\documentclass[11pt]{article}

% basic packages
\usepackage[margin=1in]{geometry}
\usepackage[pdftex]{graphicx}
\usepackage{amsmath,amssymb,amsthm}
\usepackage{custom}
\usepackage{lipsum}
\usepackage{physics}
\newcommand{\probP}{\text{I\kern-0.15em P}}
\newcommand{\comment}[1]{}

% page formatting
\usepackage{fancyhdr}
\pagestyle{fancy}
\usepackage{array}
\usepackage[open]{bookmark}
\ProvidesFile{puenc-greek.def}
\usepackage{textgreek}

\usepackage{float}



\renewcommand{\sectionmark}[1]{\markright{\textsf{\arabic{section}. #1}}}
\renewcommand{\subsectionmark}[1]{}
\lhead{\nouppercase{\rightmark}}
\chead{}
\rhead{Nuclear and Particle Physics}
\lfoot{}
\cfoot{\textbf{\thepage}}
\rfoot{}
\setlength{\headheight}{14pt}

\linespread{1.03} % give a little extra room
\setlength{\parindent}{0.2in} % reduce paragraph indent a bit
\setcounter{secnumdepth}{2} % no numbered subsubsections
\setcounter{tocdepth}{2} % no subsubsections in ToC

\begin{document}

%pedici intelligenti con " \_ "
\newcommand*\subtxt[1]{_{\textnormal{#1}}}
\DeclareRobustCommand\_{\ifmmode\expandafter\subtxt\else\textunderscore\fi}

% make title page
\thispagestyle{empty}
\bigskip \
\vspace{0.1cm}

\begin{center}
{\fontsize{22}{22} \selectfont Appunti di}
\vskip 16pt
{\fontsize{36}{36} \selectfont \bf \sffamily Nuclear and Particle Physics}
\vskip 24pt
{\fontsize{18}{18} \selectfont \rmfamily Valerio Favitta} 
\vskip 6pt
{\fontsize{14}{14} \selectfont \ttfamily vfavitta@gmail.com} 
\vskip 24pt
\end{center}

{\parindent0pt \baselineskip=15.5pt \lipsum[1-4]}

% make table of contents
\newpage
\microtoc
\newpage

% main content
\section{Introduzione}\label{sec:intro}
\import{./Sections}{1-first_sec.tex}

\section{Cinematica relativistica}
\import{./Sections}{2-second_sec.tex}

\section{Senza nome per ora}
\import{./Sections}{3-third_sec.tex}


\section{Modello statico a quark}
\import{./Sections}{4-fourth_sec.tex}

\section{Struttura degli adroni}
\import{./Sections}{5-fifth_sec.tex}

\comment{\appendix %inizio commento
\section{Revoldiv}
\begin{itemize}
    \item Lezione 1 - \autoref{sec:intro}: \href{https://revoldiv.com/posts/3e690ced-59b5-4a12-afd3-b81f8b8bc33e/}{Parte 1} - \href{https://revoldiv.com/posts/c176a40e-37f7-42c6-8d04-7b0202a01a87/}{Parte 2}
    \item Lezione 4 - \autoref{subsec:bethe-bloch}: \href{https://revoldiv.com/posts/c78e64f8-8da8-408d-a926-01382707eab4}{Parte 1} - \href{https://revoldiv.com/posts/0bc7cedb-0f13-49aa-94b2-363b8a6feb88}{Parte 2}
    \item Lezione 5 - -- : \href{https://revoldiv.com/posts/969f4a40-a80a-4e9f-aa63-0e01e39670c4}{Parte 1}
\end{itemize}
\section{Reminder}
\begin{itemize}
    \item Relazione relativistica: $\beta\gamma=\frac p E$.
    \item Fare la parte storica appena manda slide. La ho skippata parla di guerra ecc (lezione 5 e lezione 6 prima parte)
    \item 18:20:00 circa lezione 6 (44:15 della rec) parla di neutrino e roba di carica + parametro che va a zero. Non si capisce bene però sembra dica serious shit nascosta.
    \item Convenzione per i mesoni. Se consideriamo i mesoni con un quark di tipo alto, come il charm, allora esso sarà nei mesoni di carica 0 e $+1$. Ad esempio $D^+$ conterrà $c$ (e un $\bar d$ per compensare carica). Invece per quark di tipo basso è al contrario, ossia $K^+$ conterrà $\bar s$ (e un $u$), oppure $B^+$.
    \item Scoprirono prima $J/\psi$, cioè uno stato eccitato del charmonio invece del ground state $\eta_c$ perché passarono da canali elettromagnetici ossia $e^+e^-\to\gamma\to J/\psi$ e poiché $\gamma$ ha spin 1, allora anche lo stato finale deve averlo.
\end{itemize}
\section{Esercizi}
\import{./Sections}{exercise.tex}
}%fine commento
\end{document}