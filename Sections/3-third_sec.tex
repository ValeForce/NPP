\subsection{Interazioni}
\begin{itemize}
    \item L'interazione classica a distanza è descritta da un potenziale o da un campo. In meccanica quantistica invece c'è uno scambio di quanto, e ad ogni interazione è associato un bosone.
    \item Ad esempio consideriamo due cariche. Classicamente la forza che la prima carica esercita sulla seconda dipende dal campo elettrico; quantisticamente invece l'interazione tra le due cariche è mediata da un fotone, figlio della violazione di conservazione di energia in accordo con il principio di indeterminazione di Heisenberg $\Delta E\Delta t\sim\hbar$.
    \item Esistono quattro tipi di interazione: 
    \begin{enumerate}
        \item Forte. Lega i quark in adroni e protoni/neutroni in nuclei. È mediata dai gluoni.
        \item Elettromagnetica. Lega gli elettroni al nucleo formando l'atomo ed è anche responsabile delle forze molecolari in liquidi e solidi. È mediata dai fotoni.
        \item Debole. È responsabile dei decadimenti radioattivi, specie i $\beta$-decay. È mediata dai bosoni W e Z.
        \item Gravitazionale. È la più debole e riguarda ogni corpo con massa. È mediata dai gravitoni... o forse no.
    \end{enumerate}
\end{itemize}