\subsection{Interazioni}
\begin{itemize}
    \item L'interazione classica a distanza è descritta da un potenziale o da un campo. In meccanica quantistica invece c'è uno scambio di quanto, e ad ogni interazione è associato un bosone.
    \item Ad esempio consideriamo due cariche. Classicamente la forza che la prima carica esercita sulla seconda dipende dal campo elettrico; quantisticamente invece l'interazione tra le due cariche è mediata da un fotone, figlio della violazione di conservazione di energia in accordo con il principio di indeterminazione di Heisenberg $\Delta E\Delta t\sim\hbar$.
    \item Esistono quattro tipi di interazione: 
    \begin{enumerate}
        \item Forte. Lega i quark in adroni e protoni/neutroni in nuclei. È mediata dai gluoni.
        \item Elettromagnetica. Lega gli elettroni al nucleo formando l'atomo ed è anche responsabile delle forze molecolari in liquidi e solidi. È mediata dai fotoni.
        \item Debole. È responsabile dei decadimenti radioattivi, specie i $\beta$-decay. È mediata dai bosoni W e Z.
        \item Gravitazionale. È la più debole e riguarda ogni corpo con massa. È mediata dai gravitoni... o forse no.
    \end{enumerate}
    \item Come sappiamo la massa del mediatore è inversamente proporzinale al range della interazione. Se il mediatore ha massa nulla, il range è infinito (fotone ed interazione elettromagnetica). Se il mediatore ha massa finita, il range è finito. Più è massivo, più è corto il range. Infatti la interazione debole ha range molto piccolo (inferiore al fermi) e il mediatore lo metto in evidenza solo ad energie elevate.
    \item Per indicare l'intensità di ciascuna forza (non ho letto sbene slide diceva di protoni che si toccano) si pone pari a uno l'interazione forte. Allora avremo \\
    \begin{tabular}{>{\centering\arraybackslash}m{3cm} >{\centering\arraybackslash}m{3cm} >{\centering\arraybackslash}m{3cm} >{\centering\arraybackslash}m{3cm}}
        Forte & Elettromagnetica & Debole & Gravitazionale \\
        1 & $10^{-2}$    & $10^{-7}$    & $10^{-39}$    \\
    \end{tabular}\\
    Secondo Einstein forse è possibile unificare le quattro forze in un'unica teoria, ma non è ancora stato fatto. Finora solo la forza elettromagnetica e debole sono state unificate in una sola teoria. Si pensa che ad alte energie si riescono a unificare tutte le forze, solo che sono troppo elevate per raggiungerle. A $10^{16}$ GeV si uniscono forza elettromagnetica, debole e forte; A $10^{19}$ GeV si unisce anche la gravitazionale. Oggi siamo a 12 ordini di grandezza di distanza da $10^{16}$ GeV. 
    \item L'intensità di una forza è associata ad una costante di accoppiamento.
    \begin{enumerate}
        \item Quella elettromagnetica sappiamo che è la costante di struttura fine
        \begin{equation*}
            \alpha=\frac{\text{Energia elettrostatica tra due elettroni a distanza }\hbar/mc}{\text{massa a riposo dell'elettrone}}=\frac{\frac1{4\pi}\frac{e^2}{\hbar/mc}}{m_ec^2}=\frac{e^2}{4\pi\hbar c}=\frac1{137}
        \end{equation*}
        \item Quella debole interviene nei $\beta$-decay e in assorbimenti di neutrini (che sarebbe la stessa cosa). Ma interviene anche in altri processi più \textit{strani}. Vediamo i due decadimenti:
        \begin{equation*}
            \underset{dds}{\Sigma^-}\to \underset{ddu}{n}+\pi^-\qquad \tau_w\sim10^{-10}s\qquad \text{Forza debole}
        \end{equation*}
        \begin{equation*}
            \underset{uds}{\Sigma^0}\to \underset{uds}{\Lambda}+\gamma\qquad \tau\_{EM}\sim10^{-19}s\qquad \text{Forza elettromagnetica}
        \end{equation*}
    La prima è effettivamente associata all'interazione debole in quanto viene violata la conservazione di flavour (un quark strange diventa up) ed ha un tempo di $10^{-10}$s, mentre la seonda è elettromagnetica perché, oltre alla presenza di un fotone, viola soltanto la conservazione dell'isospin e il tempo è di $10^{-19}$s. Come già sappiamo, il principio di indeterminazione lega vita media e larghezza del decadimento in modo inversamente proporzionale. Ma anche la larghezza di decadimento $\Gamma$ dipende dalla costante di accoppiamento $\alpha$ che caratterizza l'interazione tra lo stato iniziale e i prodotti finali del decadimento. In generale, il tasso di decadimento (sezione d'urto) è proporzionale al quadrato della costante di accoppiamento, quindi possiamo scrivere che $\Gamma\propto\alpha^2$. Questa dipendenza dal quadrato della costante di accoppiamento è una conseguenza della teoria quantistica dei campi, dove l'ampiezza della transizione dipende linearmente da $\alpha$, mentre la probabilità di transizione (e quindi il tasso di decadimento o sezione d'urto) dipende dal modulo al quadrato di tale ampiezza. Da ciò segue che il tempo di vita medio $\tau$ è inversamente proporzionale al quadrato della costante di accoppiamento $\alpha$: $\tau\propto\alpha^{-2}$ di conseguenza, più è forte l'interazione (cioè, maggiore è la costante di accoppiamento), più breve sarà il tempo di vita della particella, mentre un'interazione più debole (con $\alpha$ più piccolo) darà luogo a un tempo di vita più lungo.
    \begin{equation*}
    \frac{\alpha_w}{\alpha}=\sqrt{\frac{\tau\_{EM}}{\tau_w}}\approx 10^{-4}-10^{-5}
    \end{equation*}
    Cioè, come già visto nella tabella precedente, la forza debole è 4-5 ordini meno intensa di quella elettromagnetica.
    \item L'interazione forte invece conserva tutto. Quindi quella elettromagnetica viola al massimo la conservazione di isospin; quella debole viola molte cose tra cui parità, coniugazione di carica, flavour ed isospin; quella forte non viola niente. Vediamo le due reazioni:
    \begin{equation*}
        \Sigma^{0*}(1385)\to\Lambda+\pi^0\qquad\tau\sim10^{-23}s\qquad{\text{Forza forte }}\Gamma=36\MeV
    \end{equation*}
    \begin{equation*}
        \Sigma^0(1192)\to\Lambda+\gamma\qquad\tau\sim10^{-19}s\qquad{\text{Forza elettromagnetica}}
    \end{equation*}
    In questo caso abbiamo 
    \begin{equation*}
    \frac{\alpha_s}\alpha=\sqrt{\frac{\tau\_{EM}}{\tau_s}}=\sqrt{\frac{10^{-19}}{10^{-23}}}\approx10^2
    \end{equation*}
    I quanti in questo caso sono i gluoni e ci sono sei cariche. La forza tra i quark è simmetrica per colore ossia non dipende da essa. Vale la cromodinamica quantistica per l'interazione forte, e si hanno i due comportamenti: 
    \\
    \begin{center}
    \begin{tabular}{>{\centering\arraybackslash}m{3cm} >{\centering\arraybackslash}m{3cm} >{\centering\arraybackslash}m{3cm}}
        Libertà asintotica & $V_s\to\alpha_s/r$ & $q^2\to\infty$\\
        Confinamento & $V_s\to kr$ & $q^2\to0$\\
    \end{tabular}\\
    \end{center}
    Cioè ad alte energie (piccole distanze) il potenziale è coulombiano, mentre a piccole energie (grandi distanze) il potenziale è elastico. La soglia di energia alta/bassa è sui GeV.
\end{enumerate}
\end{itemize}
\subsection{Diagrammi di Feynman}
