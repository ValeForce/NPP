\subsection{Interazioni}
\begin{itemize}
    \item L'interazione classica a distanza è descritta da un potenziale o da un campo. In meccanica quantistica invece c'è uno scambio di quanto, e ad ogni interazione è associato un bosone.
    \item Ad esempio consideriamo due cariche. Classicamente la forza che la prima carica esercita sulla seconda dipende dal campo elettrico; quantisticamente invece l'interazione tra le due cariche è mediata da un fotone, figlio della violazione di conservazione di energia in accordo con il principio di indeterminazione di Heisenberg $\Delta E\Delta t\sim\hbar$.
    \item Esistono quattro tipi di interazione: 
    \begin{enumerate}
        \item Forte. Lega i quark in adroni e protoni/neutroni in nuclei. È mediata dai gluoni.
        \item Elettromagnetica. Lega gli elettroni al nucleo formando l'atomo ed è anche responsabile delle forze molecolari in liquidi e solidi. È mediata dai fotoni.
        \item Debole. È responsabile dei decadimenti radioattivi, specie i $\beta$-decay. È mediata dai bosoni W e Z.
        \item Gravitazionale. È la più debole e riguarda ogni corpo con massa. È mediata dai gravitoni... o forse no.
    \end{enumerate}
    \item Come sappiamo la massa del mediatore è inversamente proporzinale al range della interazione. Se il mediatore ha massa nulla, il range è infinito (fotone ed interazione elettromagnetica). Se il mediatore ha massa finita, il range è finito. Più è massivo, più è corto il range. Infatti la interazione debole ha range molto piccolo (inferiore al fermi) e il mediatore lo metto in evidenza solo ad energie elevate.
    \item Per indicare l'intensità di ciascuna forza (non ho letto sbene slide diceva di protoni che si toccano) si pone pari a uno l'interazione forte. Allora avremo \\
    \begin{tabular}{>{\centering\arraybackslash}m{3cm} >{\centering\arraybackslash}m{3cm} >{\centering\arraybackslash}m{3cm} >{\centering\arraybackslash}m{3cm}}
        Forte & Elettromagnetica & Debole & Gravitazionale \\
        1 & $10^{-2}$    & $10^{-7}$    & $10^{-39}$    \\
    \end{tabular}
    Secondo Einstein forse è possibile unificare le quattro forze in un'unica teoria, ma non è ancora stato fatto. Finora solo la forza elettromagnetica e debole sono state unificate in una sola teoria. Si pensa che ad alte energie si riescono a unificare tutte le forze, solo che sono troppo elevate per raggiungerle. A $10^{16}$ GeV si uniscono forza elettromagnetica, debole e forte; A $10^{19}$ GeV si unisce anche la gravitazionale. Oggi siamo a 12 ordini di grandezza di distanza da $10^{16}$ GeV. 
    \item L'intensità di una forza è associata ad una costante di accoppiamento.
    \begin{enumerate}
        \item Quella elettromagnetica sappiamo che è la costante di struttura fine
        \begin{equation*}
            \alpha=\frac{\text{Energia elettrostatica tra due elettroni a distanza }\hbar/mc}{\text{massa a riposo dell'elettrone}}=\frac{\frac1{4\pi}\frac{e^2}{\hbar/mc}}{m_ec^2}=\frac{e^2}{4\pi\hbar c}=\frac1{137}
        \end{equation*}
        \item Quella debole interviene nei $\beta$-decay e in assorbimenti di neutrini (che sarebbe la stessa cosa). Ma interviene anche in altri processi più \textit{strani}. Vediamo i due decadimenti:
        \begin{equation*}
            \underset{dds}{\Sigma^-}\to \underset{ddu}{n}+\pi^-\qquad \tau\sim10^{-10}s
        \end{equation*}
        \begin{equation*}
            \underset{uds}{\Sigma^0}\to \underset{uds}{\Lambda}+\gamma\qquad \tau\sim10^{-19}s
        \end{equation*}
    \end{enumerate}
    La prima è effettivamente associata all'interazione debole in quanto viene violata la conservazione di flavour (un quark strange diventa up) ed ha un tempo di $10^{-10}$s, mentre la seonda è elettromagnetica perché, oltre alla presenza di un fotone, viola soltanto la conservazione dell'isospin e il tempo è di $10^{-19}$s. Rapporto non lo so ascolta su sez d'urto addio
\end{itemize}