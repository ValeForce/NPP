In rare occasioni vediamo \textit{violazioni} di leggi di conservazione, che sono valide solo per interazione forte ed elettromagnetica. Queste sono note come \textit{interazioni deboli} (w.i.) a causa della loro piccola costante di accoppiamento. L'interazione debole avviene in quasi tutti i processi ma il loro effetto è trascurabile, eccetto nei casi in cui tutto il resto è proibito (e.g. decadimenti che violano la stranezza, charm,\dots). A causa della interazione forte, la materia \textit{stabile} è formata solo da $u,d,e^-$. Gli altri quark e leptoni carichi sono instabili e decadono debolmente. Dunque nonostante la loro "debolezza" (piccolo range di interazione $10^{-3}$ fm e piccole sezioni d'urto $10^{-47}$ m$^2$), l'interazione forte assume un ruolo fondamentale nel determinare il nostro mondo.\\
\textit{Tutte} le particelle elementari, eccetto gluoni e fotoni (mediatori dell'interazione), vedono l'interazione debole: i quark ed i leptoni carihi interagiscono debolmente, i neutrini interagiscono \textit{solo} debolmente. Per questo motivo la nostra conoscenza dell'interazione debole, almeno fino agli anni 70, si ottenne solo da decadimenti di particelle (e.g. $\pi^+$ e $\mu^+$ che decadono) e da fasci di neutrini.
\subsubsection{Piccolo ripassino}
Rivediamo i tempi di vita media di alcuni decadimenti:
\begin{alignat*}{3}
    &\Delta^{++} \to p\pi &&\sim 10^{-23}s &&\textnormal{int. forte} \\
    &\Sigma^0\to\Lambda\gamma &&\sim6\cdot10^{-20}s \textnormal{ }&& \textnormal{int. e.m. (c'è un }\gamma) \\
    & \pi^0\to\gamma \gamma &&\sim 10^{-16}s && \textnormal{int. e.m. (ci sono due }\gamma)
\end{alignat*}
\begin{align*}
    & \begin{aligned}
       & \Sigma\to n\pi &&\sim10^{-10}s \\
       & \pi^-\to\mu\nu_\mu&&\sim10^{-8}s \\
       & \mu^-\to e^-\bar\nu_e\nu_\mu&&\sim10^{-6}s \\
       & n\to pe^-\bar\nu_e&&\sim15\textnormal{ min}
    \end{aligned}
     \left.\vphantom{\begin{matrix}
    \text{Equazione 7}\\
    \text{Equazione 8}\\
    \text{Equazione 9}\\
    \text{Equazione 10}
    \end{matrix}}\right\} \text{int. debole}
    \end{align*}
\begin{itemize}
    \item Dobbiamo spiegare l'enorme intervallo di tempi di vita media che va da $10^{-12}$ s a 15 min.
    \item L'interazione debole è anche caraterizzata da sezioni d'urto estremamente piccole \\
    ($\sim10^{-39}cm^2=1$ fb)
    \begin{alignat*}{3}
    &\sigma(\nu_\mu+N\to N+\pi+\mu)&&=10^{-38}\textnormal{cm}^2(10\,\textnormal{fb}) &&\textnormal{ ad 1 GeV}\\
    &\sigma(\pi+N\to N+\pi)&&=10^{-26}\textnormal{cm}^2(10\,\textnormal{mb}) &&\textnormal{ ad 1 GeV}
    \end{alignat*}
\end{itemize}