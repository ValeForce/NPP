In rare occasioni vediamo \textit{violazioni} di leggi di conservazione, che sono valide solo per interazione forte ed elettromagnetica. Queste sono note come \textit{interazioni deboli} (w.i.) a causa della loro piccola costante di accoppiamento. L'interazione debole avviene in quasi tutti i processi ma il loro effetto è trascurabile, eccetto nei casi in cui tutto il resto è proibito (e.g. decadimenti che violano la stranezza, charm,\dots). A causa della interazione forte, la materia \textit{stabile} è formata solo da $u,d,e^-$. Gli altri quark e leptoni carichi sono instabili e decadono debolmente. Dunque nonostante la loro "debolezza" (piccolo range di interazione $10^{-3}$ fm e piccole sezioni d'urto $10^{-47}$ m$^2$), l'interazione forte assume un ruolo fondamentale nel determinare il nostro mondo.\\
\textit{Tutte} le particelle elementari, eccetto gluoni e fotoni (mediatori dell'interazione), vedono l'interazione debole: i quark ed i leptoni carihi interagiscono debolmente, i neutrini interagiscono \textit{solo} debolmente. Per questo motivo la nostra conoscenza dell'interazione debole, almeno fino agli anni 70, si ottenne solo da decadimenti di particelle (e.g. $\pi^+$ e $\mu^+$ che decadono) e da fasci di neutrini.
\subsubsection{Piccolo ripassino}
Rivediamo i tempi di vita media di alcuni decadimenti:
\begin{alignat*}{3}
    &\Delta^{++} \to p\pi &&\sim 10^{-23}s &&\textnormal{int. forte} \\
    &\Sigma^0\to\Lambda\gamma &&\sim6\cdot10^{-20}s \textnormal{ }&& \textnormal{int. e.m. (c'è un }\gamma) \\
    & \pi^0\to\gamma \gamma &&\sim 10^{-16}s && \textnormal{int. e.m. (ci sono due }\gamma)
\end{alignat*}
\begin{align*}
    & \begin{aligned}
       & \Sigma\to n\pi &&\sim10^{-10}s \\
       & \pi^-\to\mu\nu_\mu&&\sim10^{-8}s \\
       & \mu^-\to e^-\bar\nu_e\nu_\mu&&\sim10^{-6}s \\
       & n\to pe^-\bar\nu_e&&\sim15\textnormal{ min}
    \end{aligned}
     \left.\vphantom{\begin{matrix}
    \text{Equazione 7}\\
    \text{Equazione 8}\\
    \text{Equazione 9}\\
    \text{Equazione 10}
    \end{matrix}}\right\} \text{int. debole}
    \end{align*}
\begin{itemize}
    \item Dobbiamo spiegare l'enorme intervallo di tempi di vita media che va da $10^{-12}$ s a 15 min.
    \item L'interazione debole è anche caraterizzata da sezioni d'urto estremamente piccole \\
    ($\sim10^{-39}cm^2=1$ fb)
    \begin{alignat*}{3}
    &\sigma(\nu_\mu+N\to N+\pi+\mu)&&=10^{-38}\textnormal{cm}^2(10\,\textnormal{fb}) &&\textnormal{ ad 1 GeV}\\
    &\sigma(\pi+N\to N+\pi)&&=10^{-26}\textnormal{cm}^2(10\,\textnormal{mb}) &&\textnormal{ ad 1 GeV}
    \end{alignat*}
    \item L'interazione debole viola molte leggi di conservazione: parità, coniugazione di carica, stranezza, etc.
    \item Come già detto, a causa della loro debolezza, l'interazione debole può essere osservata nella materia ordinaria solo nel decadimento $\beta$, perché non da origine ad alcun stato legato. Tuttavia, sono la base del funzionamento delle stelle che senza di essa non esisterebbe:
    \begin{equation*}
    p+p\to d+e^++\nu_e
    \end{equation*}
\end{itemize}
\subsubsection{Corrente carica e corrente neutra}
Nel modello standard, l'interazioni deboli sono classificate in due tipologie, in base alla carica del mediatore:
\begin{itemize}
    \item \textbf{Correnti cariche (CC)}, scambio di $W^\pm$: nei processi CC, la carica dei quark e leptoni cambia di $\pm1$; allo stesso tempo c'è una variazione della loro identità, cioè del flavour, secondo la teoria di Cabibbo.
    \begin{figure}[H]
        \centering
        \includegraphics[width=0.5\textwidth]{immagini/fig_cc_process.png}
        \caption{Processo di corrente carica. Un $d$ emette $W^-$ e diventa $u$.}
        %\label{fig:cc_process}
    \end{figure}
    \item \textbf{Correnti neutre (NC)}, scambio di $Z$: in questo caso i quark ed i leptoni restano invariati (non c'è FCNC Flavour-Changing Neutral Current). Fino al 1973 nessun processo NC fu osservato, anche se un esempio evidente lo abbiamo ossia il $\gamma$ della interazione elettromagnetica non porta alcuna carica.
    \item Negli anni 60 Glashow, Salam e Weinberg (e molti altri teorici) svilupparono una teoria, oggi parte del Modello Standard, che unifica la interazione debole (sia CC che NC) e l'elettromagnetismo.
    \item Il Modello Standard fu ideato \textit{prima} della scoperta di NC e del suo portatore (bosone $Z$), predetto dal Modello Standard negli anni 60 e direttamente osservato al CERN nel 1983.
\end{itemize}
\subsubsection{Classificazione}
\begin{minipage}{0.6\textwidth}
    \begin{figure}[H]
        \includegraphics[width=\textwidth]{immagini/fig_table_weak_int.png}
        %\caption{Tabella delle interazioni deboli.}
        %\label{fig:table_weak_int}
    \end{figure}
    Nell'ultima colonna l'asterisco indica che l'adrone interagente, mostrato tra le parentesi quadre, è composto. Nei diagrammi ci sono solo i quark interagenti. Gli altri partoni spettatori non partecipano alla interazione.
\end{minipage}
\begin{minipage}{0.30\textwidth}
\begin{figure}[H]
    \centering
    \includegraphics[width=\textwidth]{immagini/fig_xchange_bosons.png}
    %\caption{Scambio di bosoni nell'interazione debole.}
    %\label{fig:xchange_bosons}
\end{figure}
\end{minipage}
\subsection{Correnti cariche}
Abbiamo detto che una caratteristica dell'interazione debole è il grande range di possibili vite medie, rispetto a quello di interazione forte ed elettromagnetica che è ristretto.
\begin{table}[h!]
    \centering
    \begin{tabular}{|c|c|l|}
        \hline
        \textbf{Processo} & \textbf{Vita media (s)} & \textbf{Commento} \\
        \hline
        $\bar{\nu}_e p \to n e^+$ & infinita & I neutrini interagiscono solo debolmente (non decadono) \\
        \hline
        $n \to p e^- \bar{\nu}_e$ & $\mathcal{O}(10^3)$ & Lunga vita a causa della piccola differenza in massa p-n \\
        \hline
        $\pi^+ \to \mu^+ \nu_\mu$ & $\mathcal{O}(10^{-8})$ & I $\pi^\pm$ sono gli adroni più leggeri, dunque decadono in leptoni\\
        \hline
        $\Lambda \to p \pi^-$ & $\mathcal{O}(10^{-10})$ & Il decadimento di $\Lambda$ viola la conservazione di stranezza \\
        \hline
    \end{tabular}
    %\caption{Summary of particle decay processes.}
\end{table}
\begin{figure}[H]
    \centering
    \includegraphics[width=0.6\textwidth]{immagini/fig_decay_sketch.png}
    \caption{I decadimenti deboli più interessanti sono di solito quelli dei mesoni pesanti e.g. $K^0$ e $ B^0$.}
    %\label{fig:decay_sketch}
\end{figure}
\subsubsection{Teoria di Fermi}
\begin{itemize}
    \item La teoria moderna delle interazioni con corrente carica (una parte del Modello Standard) è il il successore della teoria di Fermi del decadimento $\beta$. 
    \item La teoria descrive l'interazione come puntuale e proporzionale alla costante d'accoppiamento $G_F$. Tuttavia, c'era un problema intrinseco in essa, ossia che non è rinormalizzabile, cioè che le sezioni d'urto divergono ad alte energie.
    \item Il Modello Standard rispetto alla teoria di Fermi va oltre l'interazione puntuale, introducendo un mediatore pesante, chiamato $W^\pm$.
    \item Il Modello Standard è matematicamente consistente, cioè è rinormalizzabile.
    \item Ma più importante è che il Modello Standard riproduce i dati sperimentali con precisione mai vista prima.
\end{itemize}
\begin{figure}[H]
    \centering
    \includegraphics[width=0.6\textwidth]{immagini/fig_fermi_to_ms.png}
    \caption{Transizione dalla teoria di Fermi al Modello Standard con i mediatori.}
    %\label{fig:fermi_to_ms}
\end{figure}
Perché il decadimento forte $n\to \pi^-p$, simile a $\Delta^0\to p\pi^-$, è proibito?
\begin{figure}[H]
    \centering
    \includegraphics[width=0.6\textwidth]{immagini/fig_n_p_pi.png}
    %\caption{Il decadimento forte $n\to \pi^-pp$ è proibito a causa della conservazione del numero barionico.}
    %\label{fig:n_p_pi}
\end{figure}
Dinamicamente è possibile, tuttavia, la differenza in massa tra neutroni e protoni è di 1.3 MeV, dunque l'unica possibile coppia fermione/antifermione con carica -1 e numero leptonico/barionico nullo è chiaramente $e^-\bar \nu_e$, perché $m(e^-)+m(\bar\nu_e)\approx m(e^-)\approx0.5$ MeV. Invece non produco pioni perché la massa del pione è 140 MeV, superiore agli 1.3 MeV a disposizione. 
\subsubsection{Costante di accoppiamento}
Ad ogni vertice di interazione nel diagramma di Feynman corrisponde una costante di accoppiamento.
\begin{itemize}
    \item Nel caso elettromagnetico la costante di accoppiamento $\alpha\propto e^2$, da cui anche la ampiezza. Invece il rate (o la sezione d'urto) proporzionale al quadrato dunque $\sigma\propto\alpha^2\propto e^4$ 
    \item Nel caso di interazioni deboli la costante di accoppiamento è $G_F$ ed è proporzionale a $g^2$, che assume il ruolo di carica della interazione debole (analoga ad $e$). Dunque l'ampiezza sarà $\propto g^2$ e $\sigma\propto g^4$.s
    \item Al contrario di $\alpha$, la $G_F$ non è adimensionata: ha le dimensioni di $E^{-2}$.
\end{itemize}
\subsection{Decadimento \texorpdfstring{$\beta$}{β}}
\[
\begin{aligned}
&n \to p + e^- + \bar{\nu}_e &\implies (A, Z) \to (A, Z+1) + e^- + \bar{\nu}_e \\
&p \to n + e^+ + \nu_e &\implies (A, Z) \to (A, Z-1) + e^+ + \nu_e \\
&p + e^- \to n + \nu_e &\implies (A, Z) + e^- \to (A, Z-1) + \nu_e
\end{aligned}
\]
\begin{itemize}
    \item L'esistenza del decadimento $\beta$ si ebbe nel 1934 con Curie.
    \item Nel 1919 Chadwick scoprì che l'elettrone proveniente dal $\beta$-decay ha uno spettro continuo.
\end{itemize}
\begin{minipage}{0.6\textwidth}
    \begin{figure}[H]
        \centering
        \includegraphics[width=\textwidth]{immagini/fig_beta_decay.png}
        \caption{Spettro continuo del decadimento $\beta$.}
        %\label{fig:beta_decay}
    \end{figure}
\end{minipage}
\begin{minipage}{0.35\textwidth}
    \begin{itemize}
        \item Il massimo in energia nello spettro corrisponde abbastanza bene al $Q$-value della reazione, mentre per il resto dello spettro c'è una violazione della conservazione di energia.
        \item Inoltre c'è una violazione della conservazione dell'impulso e del momento angolare (senza l'introduzione del neutrino).
    \end{itemize}
\end{minipage}
