In rare occasioni vediamo \textit{violazioni} di leggi di conservazione, che sono valide solo per interazione forte ed elettromagnetica. Queste sono note come \textit{interazioni deboli} (w.i.) a causa della loro piccola costante di accoppiamento. L'interazione debole avviene in quasi tutti i processi ma il loro effetto è trascurabile, eccetto nei casi in cui tutto il resto è proibito (e.g. decadimenti che violano la stranezza, charm,\dots). A causa della interazione forte, la materia \textit{stabile} è formata solo da $u,d,e^-$. Gli altri quark e leptoni carichi sono instabili e decadono debolmente. Dunque nonostante la loro "debolezza" (piccolo range di interazione $10^{-3}$ fm e piccole sezioni d'urto $10^{-47}$ m$^2$), l'interazione forte assume un ruolo fondamentale nel determinare il nostro mondo.\\
\textit{Tutte} le particelle elementari, eccetto gluoni e fotoni (mediatori dell'interazione), vedono l'interazione debole: i quark ed i leptoni carihi interagiscono debolmente, i neutrini interagiscono \textit{solo} debolmente. Per questo motivo la nostra conoscenza dell'interazione debole, almeno fino agli anni 70, si ottenne solo da decadimenti di particelle (e.g. $\pi^+$ e $\mu^+$ che decadono) e da fasci di neutrini.
\subsubsection{Piccolo ripassino}
Rivediamo i tempi di vita media di alcuni decadimenti:
\begin{alignat*}{3}
    &\Delta^{++} \to p\pi &&\sim 10^{-23}s &&\textnormal{int. forte} \\
    &\Sigma^0\to\Lambda\gamma &&\sim6\cdot10^{-20}s \textnormal{ }&& \textnormal{int. e.m. (c'è un }\gamma) \\
    & \pi^0\to\gamma \gamma &&\sim 10^{-16}s && \textnormal{int. e.m. (ci sono due }\gamma)
\end{alignat*}
\begin{align*}
    & \begin{aligned}
       & \Sigma\to n\pi &&\sim10^{-10}s \\
       & \pi^-\to\mu\nu_\mu&&\sim10^{-8}s \\
       & \mu^-\to e^-\bar\nu_e\nu_\mu&&\sim10^{-6}s \\
       & n\to pe^-\bar\nu_e&&\sim15\textnormal{ min}
    \end{aligned}
     \left.\vphantom{\begin{matrix}
    \text{Equazione 7}\\
    \text{Equazione 8}\\
    \text{Equazione 9}\\
    \text{Equazione 10}
    \end{matrix}}\right\} \text{int. debole}
    \end{align*}
\begin{itemize}
    \item Dobbiamo spiegare l'enorme intervallo di tempi di vita media che va da $10^{-12}$ s a 15 min.
    \item L'interazione debole è anche caraterizzata da sezioni d'urto estremamente piccole \\
    ($\sim10^{-39}cm^2=1$ fb)
    \begin{alignat*}{3}
    &\sigma(\nu_\mu+N\to N+\pi+\mu)&&=10^{-38}\textnormal{cm}^2(10\,\textnormal{fb}) &&\textnormal{ ad 1 GeV}\\
    &\sigma(\pi+N\to N+\pi)&&=10^{-26}\textnormal{cm}^2(10\,\textnormal{mb}) &&\textnormal{ ad 1 GeV}
    \end{alignat*}
    \item L'interazione debole viola molte leggi di conservazione: parità, coniugazione di carica, stranezza, etc.
    \item Come già detto, a causa della loro debolezza, l'interazione debole può essere osservata nella materia ordinaria solo nel decadimento $\beta$, perché non da origine ad alcun stato legato. Tuttavia, sono la base del funzionamento delle stelle che senza di essa non esisterebbe:
    \begin{equation*}
    p+p\to d+e^++\nu_e
    \end{equation*}
\end{itemize}
\subsubsection{Corrente carica e corrente neutra}
Nel modello standard, l'interazioni deboli sono classificate in due tipologie, in base alla carica del mediatore:
\begin{itemize}
    \item \textbf{Correnti cariche (CC)}, scambio di $W^\pm$: nei processi CC, la carica dei quark e leptoni cambia di $\pm1$; allo stesso tempo c'è una variazione della loro identità, cioè del flavour, secondo la teoria di Cabibbo.
    \begin{figure}[H]
        \centering
        \includegraphics[width=0.5\textwidth]{immagini/fig_cc_process.png}
        \caption{Processo di corrente carica. Un $d$ emette $W^-$ e diventa $u$.}
        %\label{fig:cc_process}
    \end{figure}
    \item \textbf{Correnti neutre (NC)}, scambio di $Z$: in questo caso i quark ed i leptoni restano invariati (non c'è FCNC Flavour-Changing Neutral Current). Fino al 1973 nessun processo NC fu osservato, anche se un esempio evidente lo abbiamo ossia il $\gamma$ della interazione elettromagnetica non porta alcuna carica.
    \item Negli anni 60 Glashow, Salam e Weinberg (e molti altri teorici) svilupparono una teoria, oggi parte del Modello Standard, che unifica la interazione debole (sia CC che NC) e l'elettromagnetismo.
    \item Il Modello Standard fu ideato \textit{prima} della scoperta di NC e del suo portatore (bosone $Z$), predetto dal Modello Standard negli anni 60 e direttamente osservato al CERN nel 1983.
\end{itemize}
\subsubsection{Classificazione}
\begin{minipage}{0.6\textwidth}
    \begin{figure}[H]
        \includegraphics[width=\textwidth]{immagini/fig_table_weak_int.png}
        %\caption{Tabella delle interazioni deboli.}
        %\label{fig:table_weak_int}
    \end{figure}
    Nell'ultima colonna l'asterisco indica che l'adrone interagente, mostrato tra le parentesi quadre, è composto. Nei diagrammi ci sono solo i quark interagenti. Gli altri partoni spettatori non partecipano alla interazione.
\end{minipage}
\begin{minipage}{0.30\textwidth}
\begin{figure}[H]
    \centering
    \includegraphics[width=\textwidth]{immagini/fig_xchange_bosons.png}
    %\caption{Scambio di bosoni nell'interazione debole.}
    %\label{fig:xchange_bosons}
\end{figure}
\end{minipage}
\subsection{Correnti cariche}
Abbiamo detto che una caratteristica dell'interazione debole è il grande range di possibili vite medie, rispetto a quello di interazione forte ed elettromagnetica che è ristretto.
\begin{table}[h!]
    \centering
    \begin{tabular}{|c|c|l|}
        \hline
        \textbf{Processo} & \textbf{Vita media (s)} & \textbf{Commento} \\
        \hline
        $\bar{\nu}_e p \to n e^+$ & infinita & I neutrini interagiscono solo debolmente (non decadono) \\
        \hline
        $n \to p e^- \bar{\nu}_e$ & $\mathcal{O}(10^3)$ & Lunga vita a causa della piccola differenza in massa p-n \\
        \hline
        $\pi^+ \to \mu^+ \nu_\mu$ & $\mathcal{O}(10^{-8})$ & I $\pi^\pm$ sono gli adroni più leggeri, dunque decadono in leptoni\\
        \hline
        $\Lambda \to p \pi^-$ & $\mathcal{O}(10^{-10})$ & Il decadimento di $\Lambda$ viola la conservazione di stranezza \\
        \hline
    \end{tabular}
    %\caption{Summary of particle decay processes.}
\end{table}
\begin{figure}[H]
    \centering
    \includegraphics[width=0.6\textwidth]{immagini/fig_decay_sketch.png}
    \caption{I decadimenti deboli più interessanti sono di solito quelli dei mesoni pesanti e.g. $K^0$ e $ B^0$.}
    %\label{fig:decay_sketch}
\end{figure}
\subsubsection{Teoria di Fermi}
\begin{itemize}
    \item La teoria moderna delle interazioni con corrente carica (una parte del Modello Standard) è il il successore della teoria di Fermi del decadimento $\beta$. 
    \item La teoria descrive l'interazione come puntuale e proporzionale alla costante d'accoppiamento $G_F$. Tuttavia, c'era un problema intrinseco in essa, ossia che non è rinormalizzabile, cioè che le sezioni d'urto divergono ad alte energie.
    \item Il Modello Standard rispetto alla teoria di Fermi va oltre l'interazione puntuale, introducendo un mediatore pesante, chiamato $W^\pm$.
    \item Il Modello Standard è matematicamente consistente, cioè è rinormalizzabile.
    \item Ma più importante è che il Modello Standard riproduce i dati sperimentali con precisione mai vista prima.
\end{itemize}
\begin{figure}[H]
    \centering
    \includegraphics[width=0.6\textwidth]{immagini/fig_fermi_to_ms.png}
    \caption{Transizione dalla teoria di Fermi al Modello Standard con i mediatori.}
    %\label{fig:fermi_to_ms}
\end{figure}
Perché il decadimento forte $n\to \pi^-p$, simile a $\Delta^0\to p\pi^-$, è proibito?
\begin{figure}[H]
    \centering
    \includegraphics[width=0.6\textwidth]{immagini/fig_n_p_pi.png}
    %\caption{Il decadimento forte $n\to \pi^-pp$ è proibito a causa della conservazione del numero barionico.}
    %\label{fig:n_p_pi}
\end{figure}
Dinamicamente è possibile, tuttavia, la differenza in massa tra neutroni e protoni è di 1.3 MeV, dunque l'unica possibile coppia fermione/antifermione con carica -1 e numero leptonico/barionico nullo è chiaramente $e^-\bar \nu_e$, perché $m(e^-)+m(\bar\nu_e)\approx m(e^-)\approx0.5$ MeV. Invece non produco pioni perché la massa del pione è 140 MeV, superiore agli 1.3 MeV a disposizione. 
\subsubsection{Costante di accoppiamento}
Ad ogni vertice di interazione nel diagramma di Feynman corrisponde una costante di accoppiamento.
\begin{itemize}
    \item Nel caso elettromagnetico la costante di accoppiamento $\alpha\propto e^2$, da cui anche la ampiezza. Invece il rate (o la sezione d'urto) proporzionale al quadrato dunque $\sigma\propto\alpha^2\propto e^4$ 
    \item Nel caso di interazioni deboli la costante di accoppiamento è $G_F$ ed è proporzionale a $g^2$, che assume il ruolo di carica della interazione debole (analoga ad $e$). Dunque l'ampiezza sarà $\propto g^2$ e $\sigma\propto g^4$.s
    \item Al contrario di $\alpha$, la $G_F$ non è adimensionata: ha le dimensioni di $E^{-2}$.
\end{itemize}
\subsection{Decadimento \texorpdfstring{$\beta$}{β}}
\[
\begin{aligned}
&n \to p + e^- + \bar{\nu}_e &\implies (A, Z) \to (A, Z+1) + e^- + \bar{\nu}_e \\
&p \to n + e^+ + \nu_e &\implies (A, Z) \to (A, Z-1) + e^+ + \nu_e \\
&p + e^- \to n + \nu_e &\implies (A, Z) + e^- \to (A, Z-1) + \nu_e
\end{aligned}
\]
\begin{itemize}
    \item L'esistenza del decadimento $\beta$ si ebbe nel 1934 con Curie.
    \item Nel 1919 Chadwick scoprì che l'elettrone proveniente dal $\beta$-decay ha uno spettro continuo.
\end{itemize}
\begin{minipage}{0.6\textwidth}
    \begin{figure}[H]
        \centering
        \includegraphics[width=\textwidth]{immagini/fig_beta_decay.png}
        %\caption{Spettro continuo del decadimento $\beta$.}
        %\label{fig:beta_decay}
    \end{figure}
\end{minipage}
\begin{minipage}{0.39\textwidth}
    \begin{itemize}
        \item Il massimo in energia nello spettro corrisponde abbastanza bene al $Q$-value della reazione, mentre per il resto dello spettro c'è una violazione della conservazione di energia.
        \item Inoltre c'è una violazione della conservazione dell'impulso e del momento angolare (senza l'introduzione del neutrino).
    \end{itemize}
\end{minipage}
\subsubsection{Il neutrino}
Per ristabilire le varie leggi di conservazione, Pauli nel 1930 ipotizzò l'esistenza del neutrino. 
\begin{itemize}
    \item La teoria di Fermi del decadimento $\beta$ fu fatta nel 1934
    \item La scoperta del neutrino avvenne nel 1958 da parte di Reines e Cowan.
    \item Abbiamo tre tipi di neutrini. Recentemente si sono scoperte le oscillazioni in flavour del neutrino che implicano che la loro massa è non nulla, sebbene sia molto piccola e ancora non misurata.
    \item Quello di Pauli era un tentativo disperato di spiegare lo spettro continuo delle emissioni $\beta$. Pauli lo chiamò "neutrone", ma poi questo nome fu dato alla particella scoperta da Chadwick nel 1932 un anno dopo. Come tutti a quell'epoca, Pauli credette che se un nucleo radioattivo emette particelle, queste devono esistere nel nucleo prima dell'emissione.
\end{itemize}
\subsubsection{Approccio di Fermi per la sua teoria}
Per formulare la sua teoria, Fermi prese come modello quello che già si sapeva ossia la QED (interazione elettromagnetica). \\
Consideriamo lo scattering elettrone-protone:\\
\begin{minipage}{0.3\textwidth}
    \begin{figure}[H]
        \centering
        \includegraphics[width=\textwidth]{immagini/fig_electron_proton_scattering.png}
        %\caption{Scattering elettrone-protone.}
        %\label{fig:electron_proton_scattering}
    \end{figure}
\end{minipage}
\begin{minipage}{0.65\textwidth}
    Dal diagramma di Feynman, l'elemento di matrice sarà
    \begin{equation*}
    M\_{fi}\approx-\frac{1}{q^2}J_\mu(e)J^\mu(p)
    \end{equation*} 
    dove $q$ è il propagatore. In generale si scrive 
    \begin{equation*}
    M\_{fi}\approx\qty(\bar u_e\sqrt\alpha\gamma^\mu u_e)\frac{g_{\mu\nu}}{q^2}\qty(\bar u_p\sqrt\alpha\gamma^\nu u_p)
    \end{equation*} 
\end{minipage}
Dunque a partire da ciò, Fermi ipotizzo che l'interazione debole fosse puntiforme: $n+\nu\to p+e^-$\\
\begin{minipage}{0.3\textwidth}
    \begin{figure}[H]
        \centering
        \includegraphics[width=\textwidth]{immagini/fig_diag_weak_int_fermi.png}
        %\caption{Interazione debole puntiforme.}
        %\label{fig:weak_interaction}
    \end{figure}
\end{minipage}
\begin{minipage}{0.65\textwidth}
    Dunque stavolta non c'è propagatore!\\
    \begin{equation*}
    M\_{fi}\approx G\qty(\bar u_p\gamma^\mu u_n)\qty(\bar{u}_e\gamma_\mu u_\nu)
    \end{equation*}
    ed è una interazione vettore-vettore perché "$\gamma^\mu$ è di dimensione 4". \\
    La costante $G$ è nota come costante di Fermi ed è legata al quadrato della \textit{carica debole}.
\end{minipage}
Si ha interazione tra due correnti cariche: si possono avere correnti adroniche e correnti leptoniche. Al solito $u_p$ distrugge la particella $p$ mentre $\bar{u}_p$ crea la particella $p$. Ad alte energie la teoria di Fermi non funziona più.
\subsubsection{Decadimento nucleare $\beta$}
La probabilità di transizione su unità di tempo si può ricavare con la regola d'oro di Fermi:
\begin{equation*}
W=\frac{2\pi}{\hbar}G^2\abs{M}^2\dv{N}{E_0}\qq{}\dv{N}{E_0}:\textnormal{ fattore dello spazio delle fasi}
\end{equation*}
Il termine $\abs{M}^2$ è il quadrato dell'elemento di matrice. Si determina integrando su tutti gli angoli delle particelle finali, sommando su tutti i possibili stati di spin finali e mediando sugli stati di spin iniziali. Il valore risultante è una costante dell'ordine di uno.\\
Con $E_0$ indichiamo l'energia disponibile nello stato finale, uguale al $Q$ della reazione. Lo spread energetico $\dd{E_0}$ è presente perché l'energia dello stato iniziale è indeterminata dal fatto che ha vita media finita (principio di indeterminazione).\\
Nel decadimento $\beta$ normalmente $E_0\approx1$ MeV. L'energia cinetica del protone è dell'ordine di $10^{-3}$ MeV e può essere trascurata. Il protone serve solo ad assicurarci che ci sia la conservzione dell'impulso. Dunque possiamo considerare suppore che l'energia finale sia suddivisa tra neutrino ed elettrone $E_0=E_e+q_\nu$.
\subsubsection{Spazio delle fasi}
\begin{itemize}
    \item Il numero di stati disponibili per un elettrone di impulso tra $p$ e $p+\dd{p}$ confinato nel volume $V$ entro un angolo solido $\dd{\Omega}$ è:
    \begin{equation*}
    \dd{N}=\frac{V\Omega}{\qty(2\pi)^3\hbar^3}p^2\dd{p}
    \end{equation*}
    \item Normalizziamo la funzione d'onda con $V=1$, e sommiamo su tutto l'angolo solido ignorando lo spin. Otteniamo
    \begin{equation*}
    \dd{N_e}=\frac{4\pi p^2\dd{p}}{\qty(2\pi)^3\hbar^3}\qq{}\dd{N_\nu}=\frac{4\pi q_\nu^2\dd{q_\nu}}{\qty(2\pi)^3\hbar^3} 
    \end{equation*}
    \item I due fattori di spazio delle fasi sono indipendenti in quanto non c'è correlazione tra $q$ e $p$, visto che nel decadimento a tre corpi il protone assorbirà la differenza di impulso. Il protone ha impulso fissato, dato da $q$ e $p$, così che il protone non ha fattore di spazio delle fasi.
    \item Il numero di stati finale è 
    \begin{equation*}
    \dd{N}=\dd{N_e}\dd{N_\nu}=\frac{\qty(4\pi)^2}{\qty(2\pi)^6\hbar^6}p^2q_\nu^2\dd{p}\dd{q_\nu}
    \end{equation*}
    \item Per un dato valore di energia dell'elettrone $E$, l'energia del neutrino $E_\nu$ p fissata, così come il suo impulso:
    \begin{equation*}
    q_\nu=E_\nu=E_0-E\implies\dd{q_\nu}=\dd{E_0}\implies\dv{N}{E_0}=\dv{N}{q_\nu}=\frac{1}{4\pi^4\hbar^6}p^2(E_0-E)^2\dd{p}
    \end{equation*}
    \item Una volta che abbiamo integrato la probabilità di transizione $W$ su tutto l'angolo solido, $\abs{M}^2$ è uguale ad una costante, dunque lo spettro energetico dell'elettrone dipende soltanto dal fattore di spazio delle fasi:
    \begin{equation*}
    N(p)\dd{p}\propto p^2(E_0-E)^2\dd{p}
    \end{equation*}
\end{itemize}
\subsubsection{Kurie plot}
\begin{itemize}
    \item Se facciamo il grafico di $\qty(\frac{N(p)}{p^2})^{\frac{1}{2}}$ in funzione dell'energia dell'elettrone, otteniamo una retta che interseca l'asse $x$ in $E=E_0$. Questo plot fu sviluppato da Kurie.
    \item Sperimentalmente dobbiamo includere un fattore correttivo $F(Z,p)$ per tenere conto dell'interazione coulombiana tra elettrone e nuclei.
    \item Se il neutrino avesse massa, i suoi effetti modificherebbero la distribuzione:
    \begin{equation*}
    N(p)\dd{p}\propto p^2(E-E_0)^2\sqrt{1-\qty(\frac{m_\nu}{E_0-E})^2}\dd{p}
    \end{equation*}
    \item Il plot di Kurie è modificato in modo che la curva intersechi l'asse $x$ in $E=E_0-m_\nu$. Questo è un tentativo di misurare la massa del neutrino elettronico... Sfortunatamente in questa regione ci sono pochi eventi per difficoltà sperimentali/tecniche. Al momento abbiamo solo un limite superiore
    \begin{equation*}
    m_\nu\leq2.2 \eV
    \end{equation*}
\end{itemize}
\subsubsection{La regola di Sargent}
\begin{itemize}
    \item Il rate totale di decadimento si ottiene integrando lo spettro $N(p)\dd{p}$. Si può fare analiticamente, tuttavia considerando l'elettrone relativisticamente (cioè quasi sempre) possiamo approssimare $p\approx E$ ed ottenere una formula semplice:
    \begin{equation*}
    N\propto\int_{0}^{E_0}E^2(E_0-E)^2\dd{E}\propto E_0^5
    \end{equation*}
    \item Il rate di decadimento è proporzionale alla quinta potenza dell'energia disponibile nel processo. Questa è la regola di Sargent. 
    \item Considerando tutti i fattori numerici nel processo, otteniamo 
    \begin{equation*}
        W=\frac{G^2\abs{M}^2E_0^5}{60\pi^3(\hbar c)^6\hbar}\qq{Per }E_0\gg m_e
    \end{equation*}
    \item La costante di Fermi $G$ si può ricavare da misure di tempo di vita media di decadimenti $\beta$ (e qualche speculazione teorica, vedi angoli di Cabibbo) o in modo più preciso dalla vita media del $\mu$.
    \item Dalla PDG otteniamo $\frac{G}{(\hbar c)^3}=1.16637(1)\cdot10^{-5}\textnormal{ GeV}^{-2}$
    \item Sostituendo valori numerici otteniamo $\frac{1}{\tau}=W=\frac{1.11}{10^4}\abs{M}^2E_0^5s^{-1}$\qq{($E_0$ in MeV)}
    \item Per esempio se $E_0\approx100$ MeV come nel decadimento del muone e $\abs{M}^2=1$, otteniamo
    \begin{equation*}
    \tau_\mu=\frac{1}{W}\approx10^{-6}s\qq{$\tau_\mu=2.2\mu$s}
    \end{equation*}
    \item La dipendenza da $E_0^5$ spiega il grande range di tempi di vita media dei decadimenti deboli.
\end{itemize}
\subsubsection{Decadimento nucleare $\beta$}
\begin{itemize}
    \item Il decadimento $\beta$ si distingue in transizioni permesse e proibite.
    \item Quelle permesse sono le più comuni e caratterizzate da lfatto che l'elettrone ed il neutrino emessi non portano al cun momento angolare spaziale, cioè siamo in uno stato $s$ ($L=0$). Questo è giustificato da lfatto che i due leptoni hanno energie di circa un MeV.
    \item La transizione con $L=1$ è detta \textit{prima proibita}, con $L=2$ \textit{seconda proibita}, etc. Hanno vite medie molto più lunghe di quella permessa.
    \item Poiché $e$ e $\nu$ hanno spin 1/2, la variazione di spin del nucleo può essere 0 o 1. Le transizioni con $\Delta J=0$ sono dette di Fermi, con $\Delta J=1$ di Gamow-Teller.
    \item Poiché elettrone e neutrino hanno $L=0$, non c'è alcun cambio nel momento angolare del nucleo, dunque la parità non varia. Il nucleo esegue uno spin flip per transizioni di Gamow-Teller.
\end{itemize}
\subsubsection{La costante di Fermi}
\begin{itemize}
    \item Il rate di decadimento può essere scritto in modo differente rispetto alla regola di Sargent. Scriviamo esplicitamente la massa del protone, includiamo il fattore dello spazio delle fasi e la correzione coulombiana $F(Z,p)$ in una funzione adimensionata $f(\pm Z,E_0/m_e)$ che può essere calcolata analiticamente.
    \begin{align*}
    &\frac{1}{\tau}=W=\frac{(mc^2)^5}{2\pi^3\hbar(\hbar c)^6}G^2\abs{M}^2f(\pm Z,E_0)\\
    & G^2\abs{M}^2=\frac{\textnormal{cost.}}{f\tau}\qq{(cost.=)}\frac{2\pi^3}{m^5}
    \end{align*}
    \item Al contrario delle grandi variazioni di vita medie a causa della forte dipendenza di $f$ da $p_e^{\textnormal{max}}$, il prodotto $G^2\abs{M}^2$ è circa uguale in tutti i decadimenti.
    \item Tuttavia si osserva una piccola differenza nel tipo di transizione nucleare: Fermi, Gamow-Teller o miste.
    \item Se consideriamo una transizione di Fermi pura, otteniamo
    \begin{equation*}
    \frac{G}{(\hbar c)^3}=1.140(2)\cdot10^{-5}\textnormal{ GeV}^{-2}
    \end{equation*}
    che è leggermente diversa da quella ottenuta dal $\mu$ dalla PDG. Vedremo più avanti il motivo di questa discrepanza (angoli di Cabibbo).
\end{itemize}