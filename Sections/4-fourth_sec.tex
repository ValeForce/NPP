pippo
\subsection{Simmetrie e leggi di conservazione}
In generale una legge fisica è simmetrica rispetto ad una trasformazione quando la forma della legge è invariante per questa trasformazione. Sia in meccanica classica che quantistica. 
\begin{itemize}
    \item In QM se l'operatore non dipende esplicitamente dal tempo allora commuta con la hamiltoniana del sistema. In generale (non sempre?) i numeri quantici conservati sono associati ad operatori che commutano con l'hamiltoniana.
    \item Le simmetrie si dividono in continue e discrete. Vediamo ad esempio la traslazione spaziale (continua)
    \begin{equation*}
        \psi(r+\delta r)=\psi(r)+\delta r \dv{\psi}{r}=\qty(1+\delta r \pdv{}{r})\psi(r)
    \end{equation*}
    L'operatore che descrive una traslazione finita è l'impulso:
    \begin{equation*}
        D=1+\frac i\hbar p\delta r\implies D=\lim_n\qty(1+\frac {ip\Delta r}{n\hbar})^n=e^{\frac i\hbar p\Delta r}
    \end{equation*}
    Chiamiamo $p$ generatore dell'operatore $D$ di traslazione spaziale. Se l'hamiltoniana è invariante per traslazioni, allora commuta con $D$ e dunque anche con $p$. Questo lo si può esprimere in tre modi equivalenti:
    \begin{enumerate}
        \item L'impulso si conserva in un sistema isolato.
        \item L'hamiltoniana è invariante per traslazioni spaziali.
        \item L'operatore impulso commuta con l'hamiltoniana.
    \end{enumerate}
    \item A simmetrie continue associamo numeri quantici additivi, a simmetrie discrete sono associati numeri quantici moltiplicativi.
\end{itemize}
\subsubsection{Parità}
La trasformazione di parità è l'inversione delle coordiante spaziali. 
\begin{equation*}
    P\psi(\vec r)=\psi(-\vec r)
\end{equation*}
Chiaramente se applico due volte l'operatore ottengo la funzione iniziale:
\begin{equation*}
P^2\psi(\vec r)=\psi(\vec r)\implies P^2=1 \text{ (unitario) }\implies \lambda=\pm 1
\end{equation*}
Vediamo degli esempi:
\begin{itemize}
    \item Consideriamo le semplici funzioni trigonometriche 
    \begin{gather*}
        \psi(x)=\cos x \overset{P}{\longrightarrow}\cos(-x)=\cos(x)=\psi(x)\text{ Pari} \\
        \psi(x)=\sin x \overset{P}{\longrightarrow}\sin(-x)=\sin(x)=-\psi(x)\text{ Dispari} 
    \end{gather*}
    In generale una combinazione lineare $\psi(x)=\cos x+\sin x$ non è detto che sia simmetrica per parità.
    \item Un altro esempio può essere la funzione d'onda di un elettrone in un atomo di idrogeno. La simmetria della funzione d'onda ha la stessa parità di $l$. Infatti
    \begin{equation*}
        \psi(r,\vartheta,\varphi)=\chi(r)\sqrt{\frac{(2l+1)(l-m)!}{4\pi(l+m)!}}P_m^l(\cos\vartheta)e^{im\varphi}
    \end{equation*}
    Fare una trasformazione di parità vuol dire 
    \begin{equation*}
    \vec r\to-\vec r\implies 
    \begin{cases}
        \vartheta\to\pi-\vartheta\\
        \varphi\to\pi+\varphi
    \end{cases}\implies 
    \begin{cases}
    e^{im\varphi}\to e^{im(\pi+\varphi)}=(-1)^me^{im\varphi}\\
    P_m^l(\cos\vartheta)\to (-1)^{l+m}P_m^l(\cos\vartheta)
    \end{cases}\implies 
    \end{equation*}
    \begin{equation*}
    \implies Y_l^m(\vartheta,\varphi)\to (-1)^{l+2m}Y_l^m(\vartheta,\varphi)=(-1)^lY_l^m(\vartheta,\varphi)
    \end{equation*}
    Questo risultato vale in generale per le armoniche sferiche, che quindi hanno parità data da $l$. Nelle transizioni di dipolo elettrico la regola di selezione è $\Delta l=\pm 1$, quindi la parità atomica cambia. Però nei processi elettromagnetici la parità si conserva, quindi la parità della radiazione emessa deve essere negativa per compensare la parità. 
    \item In questo caso il numero quantico è moltiplicativo e non si conserva soltanto nei decadimenti deboli. Inoltre è necessario che per convenzione assegniamo una parità intrinseca a ciascuna particella: a protoni e neutroni assegniamo parità positiva.
    \item Assegnare la parità intrinseca serve a distinguere particelle che interagiscono tra di loro (come cariche elettriche). Chiaramente il segno della parità intrinseca è scelto arbitrariamente, quello che conta è la parità relativa tra due particelle. Ad esempio particelle ed antiparticelle hanno parità opposta. Ad esempio nella reazione per la scoperta dell'antiprotone $p+p\to p+p+\bar p+p$ la parità totale nel canale di ingresso è uguale a quella in uscita (l'interazione forte la conserva). Questo discorso però funziona solo per i fermioni! Nel caso di fermioni, particella ed antiparticella hanno la stessa parità. 
    \item I vettori polari cambiano segno sotto trasformazione di parità e quelli assiali (pseudovettori) no.
    \begin{equation*}
    \begin{cases}
    \vec r\to-\vec r\\
    \vec p\to-\vec p\\
    \vec E\to-\vec E\\
    \end{cases}\text{ Polari}\qquad
    \begin{cases}
        \vec \sigma\to\vec \sigma\\
        \vec L\to\vec L\\
        \vec B\to\vec B\\
    \end{cases}\text{ Assiali}
    \end{equation*}
\end{itemize}
\subsubsection{Parità del pione carico}
Consideriamo il decadimento $\pi^-+d\to n+n$, che è un processo forte perché tutto si conserva.\\
Canale di ingresso: $\pi^-+d$
\begin{enumerate}
    \item Momento angolare totale $j$ iniziale:
    \begin{itemize}
        \item Supponiamo che lo stato sia preparato sperimentalmente con $l=0$.
        \item La particella $\pi^-$ ha spin $s=0$, quindi non contribuisce con il proprio spin. Inoltre ha parità negativa (è un cosiddetto mesone pseudoscalare).
        \item Il deuterone (d) ha $s=1$ e parità positiva (essendo un sistema di due nucleoni in uno stato legato con parità).
    \end{itemize}
    Pertanto il momento angolare iniziale è $j=1$. 
    \item Parità iniziale:
    \begin{itemize}
        \item La parità del sistema $\pi^-+d$ è data dalla parità del prodotto tra il $\pi^-$ e il deuterone:
        \begin{equation*}
        P\_{ingresso}=P_{\pi^-}P_d=(-1)\cdot (+1)= -1
        \end{equation*}
    \end{itemize}
\end{enumerate}
Canale di uscita: $n+n$
\begin{enumerate}
    \item Momento angolare totale $j$ finale:
    \begin{itemize}
        \item Gli stati possibili dei neutroni sono $s=0$ (singoletto) e $s=1$ (tripletto).
        \item Il momento angolare orbitale $l$ tra i due neutroni determinerà il valore di $j$ nel canale di uscita:
        \begin{equation*}
        j=l\pm s
        \end{equation*}
        \item Per i neutroni, che sono fermioni, il sistema complessivo deve essere antisimmetrico. Questo implica che se $s=0$ (singoletto), $l$ deve essere pari; se $s=1$ (tripletto), $l$ deve essere dispari.
    \end{itemize} 
    \item Parità finale (canale di uscita):
    \begin{itemize}
        \item La parità del sistema $n+n$ è data da:
        \begin{equation*}
        P\_{uscita}=(-1)^l
        \end{equation*}
        visto che i neutroni hanno parità $+1$.
    \end{itemize}
\end{enumerate}
\comment{Lo stato iniziale ha $l=0$ (preparato sperimentale) con $s_\pi=0$ e $s_d=1$ allora $j=1$. Dunque anche lo stato finale dovrà avere $j=1$. 
    La parità dello stato finale è data da 
    \begin{equation*}
    K=\underbrace{(-1)^{s+1}}_{\text{spin}}\underbrace{(-1)^l}_{\text{orbitale}}=(-1)^{l+s+1}
    \end{equation*}
    poiché sono fermioni}%fine commento
Allora per conservazione di momento angolare e parità, dovrò avere per gli stati finali 
\begin{gather*}
j\_i=j\_f\implies 1=l+s\\
P\_{ingresso}=P\_{uscita}\implies -1=(-1)^l\implies l \text{ dispari}
\end{gather*}
Visto che $s=0,1$ e $l$ deve essere dispari, allora $l=1$ e $s=1$ è l'unica combinazione che soddisfa entrambe le condizioni. Quindi il canale di uscita $n+n$ ha $l=1$ e $s=1$ (tripletto).
\subsubsection{Parità del pione neutro}
Il pione neutro decade con BR $=99\%$ in $\gamma+\gamma$. Questo decadimento è un processo elettromagnetico, quindi la parità si conserva. 
\begin{itemize}
\item Siano $\vec k$ e $-\vec k$ gli impulsi dei due fotoni e $\varepsilon_1$ e $\varepsilon_2$ i vettori polarizzazione.
\item Visto che i fotoni sono bosoni, la funzione d'onda totale deve essere simmetrica per scambio di particelle.
\begin{gather*}
\psi\text{ pari: } \psi_1(2\gamma)=A(\vec\varepsilon_1\cdot\vec\varepsilon_2)\propto\cos\varphi\\
\psi\text{ dispari: } \psi_1(2\gamma)=B(\vec\varepsilon_1\times\vec\varepsilon_2)\cdot \vec k\propto\sin\varphi
\end{gather*}
dove $\varphi$ è l'angolo tra i piani di polarizzazione dei fotoni. La $\psi_1$ è scalare (parità positiva), la $\psi_2$ è pseudoscalare (parità negativa). 
\item Quindi per ora non sappiamo quale delle due funzioni d'onda è quella giusta. Abbiamo i due casi 
\begin{gather*}
P_{\pi^0}=+1\implies \abs{\psi}^2\propto\cos^2\varphi\\
P_{\pi^0}=-1\implies \abs{\psi}^2\propto\sin^2\varphi
\end{gather*}
Per misurare devo studiare il decadimento del $\pi^0\to\gamma\gamma$. Noi ovviamente per rivelare i fotoni andiamo a cercare le coppie elettrone-positrone. Dunque cerchiamo 
\begin{equation*}
\pi^0\to\gamma\gamma\to e^+e^-e^+e^-
\end{equation*}
detto decadimento \textit{doppio Dalitz} con BR $(3.14\pm0.30)\cdot10^{-5}$. Da un grafico (che metterò mai visto che non passa le slide) vedo distribuzione sperimentale che è in accordo con andamento con parità negativa (funzione d'onda va con $\sin^2\varphi$).
\end{itemize}
\subsubsection{Conservazione della parità}
Abbiamo detto che la parità è conservata nei processi forti ed elettromagnetici e non nei processi deboli. 
\begin{itemize}
    \item Un esempio è il neutrino. Noi li conosciamo solo sinistri (left-handed) e non destri. Infatti un neutrino come lo conosciamo noi ha elicità negativa, cioè impulso e spin antiparalleli. Se facciamo una trasformazione di parità solo l'impulso cambia segno e non lo spin, quindi si avrebbe un \textit{neutrino destro} che in realtà non esiste, o meglio non interagisce con la materia attraverso le forze che conosciamo presenti nel Modello Standard, dunque non è rivelabile. 
    \item In realtà sperimentalmente è stata misurata una piccola violazione della parità in processi forti ed elettromagnetici. Ciò è dovuto al fatto che la hamiltoniana in realtà è composta da tre pezzi delle tre interazioni, quindi la parità non è conservata in generale perché c'è sempre un piccolo contributo di interazione debole. Ad esempio questo problema non sorge con l'elettrone, che sappiamo esistere sia destro che sinistro in quanto ha comunque la carica elettrica e quindi mal che vada elettromagneticamente lo riveliamo sempre. Il problema sussiste solo con il neutrino.
    \item Nella corrente carica, cioè scambio di $W^\pm$, la parità è violata al 100\%. Questo perché il $W$ interagisce sempre solo con particella sinistra o antiparticella destra (ignora le altre due tipologie, è razzista, non ci interagisce). Nel caso della corrente neutra, cioè scambio di $Z^0$, la situazione è più complessa. Infatti esso interagisce sia con particelle sinistre che destre, ma con pesi (accoppiamenti) diversi, di meno con particelle destre. Invece interazione forte ed elettromagnetica non distinguono elicità, cioè particelle destre o sinistre, e quindi la parità è conservata. 
\end{itemize}
\subsubsection{Particelle ed antiparticelle}
\begin{itemize}
    \item Al solito noi ci aspettiamo che esistono antiparticelle anche solo dalla relatività speciale in quanto ci sono soluzioni con energia negativa. 
    \item In meccanica quantistica rappresentiamo l'ampiezza di un flusso di particelle (e.g. elettroni) come una funzione d'onda piana
    \begin{equation*}
        \psi(x)=Ae^{\frac i\hbar(px-E t)}
    \end{equation*}
    questa espressione rappresenta anche particelle di energia $-E$ e impulso $-p$ che si muovono in direzione opposta nello spazio e nel tempo (anche in Klein-Gordon, per questo nei diagrammi di Feynman hanno direzione indietro nel tempo).
\end{itemize}
\subsubsection{Coniugazione di carica}
\begin{itemize}
    \item L'effetto dell'operatore coniugazione di carica $C$ è di invertire carica e il momento magnetico della particella.
    \item In fisica classica abbiamo che le leggi di Maxwell sono invarianti per esse, che sono 
    \begin{align*}
        q&\to-q\\
        \vec J&\to-\vec J\\
        \vec E&\to -\vec E\\
        \vec H&\to -\vec H
    \end{align*}
    \item In meccanica quantistica invece quando applichiamo $C$ abbiamo\\ 
    \begin{center}
    \begin{tabular}{>{\centering\arraybackslash}m{3cm} >{\centering\arraybackslash}m{3cm} >{\centering\arraybackslash}m{3cm}}
          & Protone & Antiprotone \\
        Carica & +e & -e \\
        N. barionico & 1 & -1 \\
        Momento magnetico & $\frac{e\hbar}{2mc}=2.79$ & -2.79 \\
        Spin & $\frac12\hbar$ & $-\frac12\hbar$ \\
    \end{tabular}
    \end{center}
    \item Consideriamo il neutrino sinistro. Se effettuiamo la trasformazione di parità otteniamo un neutrino destro (spin e impulso parallelo); se effettuiamo la trasformazione di coniugazione di carica otteniamo un antineutrino sinistro (spin e impulso antiparalleli). Entrambe particelle che \textit{non} esistono. Tuttavia se effettuiamo entrambe le trasformazioni, otteniamo un antineutrino destro, che è una particella che esiste. Quindi c'è buona simmetria $CP$ (o $PC$) che viene rispettata... c'è solo una piccolissima violazione. 
\end{itemize}
\subsubsection{Autostati dell'operatore $C$}
Solo i bosoni neutri che coincidono con la propria antiparticella possono essere autostati di $C$.
\begin{itemize}
\item Se applichiamo $C$ ad un pione carico non otteniamo un autostato:
\begin{equation*}
C\ket{\pi^+}=\ket{\pi^-}\neq\lambda\ket{\pi^+}
\end{equation*}
dunque i pioni carichi non sono autostati di $C$. 
\item Invece abbiamo, poichè applicando due volte $C$ torniamo allo stato di partenza $C^2\ket{\pi^0}=\ket{\pi^0}$,
\begin{equation*}
C\ket{\pi^0}=\lambda\ket{\pi^0}\implies C^2\ket{\pi^0}=\lambda^2\ket{\pi^0}=\ket{\pi^0}\implies \lambda=\pm 1
\end{equation*}
Per determinare se è positivo o negativo, ricordiamo che il decadimento è $\pi^0\to\gamma\gamma$ quindi $C(\pi^0)=+1$ perché l'interazione elettromagnetica è invariante per coniugazione di carica (la conserva). Lo si capisce anche classicamente, infatti cambiando la carica cambiano segno sia $\vec E$ sia $\vec B$. 
\item Dunque $C(\gamma)=-1$ e da ciò deduciamo che il decadimento $\pi^0\to3\gamma$ è proibito. D'altra parte troviamo sperimentalmente
\begin{equation*}
\frac{\text{BR}(\pi^0\to3\gamma)}{\text{BR}(\pi^0\to2\gamma)}<3.1\times10^{-8}
\end{equation*}
\end{itemize}
\subsubsection{Conservazione di $C$}
\begin{itemize}
\item La coniugazione di carica si conserva in interazioni forti ed elettromagnetiche. Allora nei processi forti mi aspetto sempre particelle ed antiparticelle per compensarsi.
\item Il mesone $\eta$ decade in vari modi ($J^P=0^-, m=550\MeV$).
\begin{enumerate}
    \item $\eta\to\gamma\gamma$ BR $=$ 39.4\%
    \item $\eta\to\pi^+\pi^-\pi^0$ BR $=$ 23.1\%
    \item $\eta\to\pi^+\pi^-\gamma$ BR $=$ 4.7\%
    \item $\eta\to\pi^0e^+e^-$ BR $<4\times10^{-5}$\%
\end{enumerate} 
poiché $\eta\to\gamma\gamma$ è il decadimento principale, si deve avere $C(\eta)=+1$. Dunque il decadimento $\eta\to\pi^0e^+e^-$ è proibito dalla conservazione di C. Ma questo è dovuto al fatto che $C(e^+e^-)=+1$. Vediamo perché.
\end{itemize}
\subsubsection{Decadimento del positronio}
\begin{itemize}
\item Il positronio è uno stato legato $e^+e^-$ che possiede livelli energetici simili all'atomo di idrogeno (con circa la metà dello spazio). La funzione d'onda la suddividiamo:
\begin{equation*}
\psi(e^+e^-)=\varphi(\text{spazio})\times\alpha(\text{spin})\times \chi(\text{carica})
\end{equation*}
\begin{enumerate}
    \item $\varphi(\text{spazio})$. Lo scambio di particelle è equivalente all'inversione spaziale, dunque questo termine introduce solo un fattore $(-1)^L$ dove $L$ è il momento angolare orbitale.
    \item $\alpha(\text{spin})$. Avendo due fermioni, si possono accoppiare in singoletto ($s=0$) o tripletto ($s=1$). Indicando gli stati con $\psi(s,s_z)$, abbiamo:
    \begin{equation*}
        \begin{cases}
            \alpha(1,1)=\psi_1(\frac12,\frac12)\psi_2(\frac12,\frac12)\\
            \alpha(1,0)=\frac1{\sqrt2}\qty(\psi_1(\frac12,\frac12)\psi_2(\frac12,-\frac12)+\psi_1(\frac12,-\frac12)\psi_2(\frac12,\frac12))\\
            \alpha(1,-1)=\psi_1(\frac12,-\frac12)\psi_2(\frac12,-\frac12)
        \end{cases}\quad
        \begin{array}{c}
            s=1\text{ Tripletto}\\
            \text{Simmetrico}
         \end{array} 
    \end{equation*}
    \begin{equation*}
        \alpha(0,0)=\frac1{\sqrt2}\qty(\psi_1\qty(\frac12,\frac12)\psi_2\qty(\frac12,-\frac12)-\psi_1\qty(\frac12,-\frac12)\psi_2\qty(\frac12,\frac12))\quad\begin{array}{c}
            s=0\text{ Singoletto}\\
            \text{Antisimmetrico}
         \end{array}
    \end{equation*}
    La simmetria della parte di spin è data da $(-1)^{s+1}$, dunque è pari se $s$ è dispari e viceversa.
    \item Per la parte carica invece consideriamo un fattore $C$.
\end{enumerate}
\item Dunque la simmetria totale della funzione d'onda per scambio di elettroni e positroni è 
\begin{equation*}
    K=(-1)^L(-1)^{s+1}C
\end{equation*}
siccome abbiamo un sistema di due fermioni, la simmetria totale deve essere antisimmetrica $K=-1$.
\item Allora con $L=0(\implies J=s)$ sperimentalmente posso avere due diversi decadimenti
\begin{equation*}
    e^+e^-\to2\gamma\qquad e^+e^-\to 3\gamma
\end{equation*}
e dunque possiamo avere (il vincolo è $K=-1=(-1)^{L+s+1}C$)
\begin{center}
    \begin{tabular}{>{\centering\arraybackslash}m{2.5cm} >{\centering\arraybackslash}m{2.5cm} >{\centering\arraybackslash}m{2.5cm}>{\centering\arraybackslash}m{2.5cm}>{\centering\arraybackslash}m{2.5cm}}
          & $J=S$ & $L$ & $C$ & $K$ \\
        $2\gamma$ & 0 & 0 &+1 &-1  \\
        $3\gamma$ & 1 & 0 & -1&-1 \\
    \end{tabular}
    \end{center}
perché sappiamo già che $C(n\gamma)=(-1)^n$.
\item Le larghezze di decadimento si possono confrontare teoricamente dalla QED e sperimentalmente. C'è accordo!
\begin{center}
    \begin{tabular}{>{\centering\arraybackslash}m{1cm} >{\centering\arraybackslash}m{4cm} >{\centering\arraybackslash}m{4cm}>{\centering\arraybackslash}m{4cm}}
          & $\Gamma$ & $\tau$ (teoria) & $\tau$ (esperimenti) \\
          $2\gamma$ & $\frac12mc^2\alpha^5$ & $1.252\times10^{-10}$ s &$\qty(1.252\pm0.017)\times 10^{-10} $s   \\
          $3\gamma$ & $\frac2{9\pi}(\pi^2-9)\alpha^6mc^2$ & $1.374\times10^{-7}$ s& $\qty(1.377\pm0.004)\times10^{-7}$ s \\
    \end{tabular}
    \end{center}
    Notiamo che entrambi sono possibili! 24:49 fa roba freestyle.
\end{itemize}