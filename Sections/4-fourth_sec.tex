pippo
\subsection{Simmetrie e leggi di conservazione}
In generale una legge fisica è simmetrica rispetto ad una trasformazione quando la forma della legge è invariante per questa trasformazione. Sia in meccanica classica che quantistica. 
\begin{itemize}
    \item In QM se l'operatore non dipende esplicitamente dal tempo allora commuta con la hamiltoniana del sistema. In generale (non sempre?) i numeri quantici conservati sono associati ad operatori che commutano con l'hamiltoniana.
    \item Le simmetrie si dividono in continue e discrete. Vediamo ad esempio la traslazione spaziale (continua)
    \begin{equation*}
        \psi(r+\delta r)=\psi(r)+\delta r \dv{\psi}{r}=\qty(1+\delta r \pdv{}{r})\psi(r)
    \end{equation*}
    L'operatore che descrive una traslazione finita è l'impulso:
    \begin{equation*}
        D=1+\frac i\hbar p\delta r\implies D=\lim_n\qty(1+\frac {ip\Delta r}{n\hbar})^n=e^{\frac i\hbar p\Delta r}
    \end{equation*}
    Chiamiamo $p$ generatore dell'operatore $D$ di traslazione spaziale. Se l'hamiltoniana è invariante per traslazioni, allora commuta con $D$ e dunque anche con $p$. Questo lo si può esprimere in tre modi equivalenti:
    \begin{enumerate}
        \item L'impulso si conserva in un sistema isolato.
        \item L'hamiltoniana è invariante per traslazioni spaziali.
        \item L'operatore impulso commuta con l'hamiltoniana.
    \end{enumerate}
    \item A simmetrie continue associamo numeri quantici additivi, a simmetrie discrete sono associati numeri quantici moltiplicativi.
\end{itemize}
\subsubsection{Parità}
La trasformazione di parità è l'inversione delle coordiante spaziali. 
\begin{equation*}
    P\psi(\vec r)=\psi(-\vec r)
\end{equation*}
Chiaramente se applico due volte l'operatore ottengo la funzione iniziale:
\begin{equation*}
P^2\psi(\vec r)=\psi(\vec r)\implies P^2=1 \text{ (unitario) }\implies \lambda=\pm 1
\end{equation*}
Vediamo degli esempi:
\begin{itemize}
    \item Consideriamo le semplici funzioni trigonometriche 
    \begin{gather*}
        \psi(x)=\cos x \overset{P}{\longrightarrow}\cos(-x)=\cos(x)=\psi(x)\text{ Pari} \\
        \psi(x)=\sin x \overset{P}{\longrightarrow}\sin(-x)=\sin(x)=-\psi(x)\text{ Dispari} 
    \end{gather*}
    In generale una combinazione lineare $\psi(x)=\cos x+\sin x$ non è detto che sia simmetrica per parità.
    \item Un altro esempio può essere la funzione d'onda di un elettrone in un atomo di idrogeno. La simmetria della funzione d'onda ha la stessa parità di $l$. Infatti
    \begin{equation*}
        \psi(r,\vartheta,\varphi)=\chi(r)\sqrt{\frac{(2l+1)(l-m)!}{4\pi(l+m)!}}P_m^l(\cos\vartheta)e^{im\varphi}
    \end{equation*}
    Fare una trasformazione di parità vuol dire 
    \begin{equation*}
    \vec r\to-\vec r\implies 
    \begin{cases}
        \vartheta\to\pi-\vartheta\\
        \varphi\to\pi+\varphi
    \end{cases}\implies 
    \begin{cases}
    e^{im\varphi}\to e^{im(\pi+\varphi)}=(-1)^me^{im\varphi}\\
    P_m^l(\cos\vartheta)\to (-1)^{l+m}P_m^l(\cos\vartheta)
    \end{cases}\implies 
    \end{equation*}
    \begin{equation*}
    \implies Y_l^m(\vartheta,\varphi)\to (-1)^{l+2m}Y_l^m(\vartheta,\varphi)=(-1)^lY_l^m(\vartheta,\varphi)
    \end{equation*}
    Questo risultato vale in generale per le armoniche sferiche, che quindi hanno parità data da $l$. Nelle transizioni di dipolo elettrico la regola di selezione è $\Delta l=\pm 1$, quindi la parità atomica cambia. Però nei processi elettromagnetici la parità si conserva, quindi la parità della radiazione emessa deve essere negativa per compensare la parità. 
    \item In questo caso il numero quantico è moltiplicativo e non si conserva soltanto nei decadimenti deboli. Inoltre è necessario che per convenzione assegniamo una parità intrinseca a ciascuna particella: a protoni e neutroni assegniamo parità positiva.
\end{itemize}
\subsubsection{Parità del pione carico}
Consideriamo il decadimento $\pi^-+d\to n+n$, che è un processo forte perché tutto si conserva.\\
Canale di ingresso: $\pi^-+d$
\begin{enumerate}
    \item Momento angolare totale $j$ iniziale:
    \begin{itemize}
        \item Supponiamo che lo stato sia preparato sperimentalmente con $l=0$.
        \item La particella $\pi^-$ ha spin $s=0$, quindi non contribuisce con il proprio spin. Inoltre ha parità negativa (è un cosiddetto mesone pseudoscalare).
        \item Il deuterone (d) ha $s=1$ e parità positiva (essendo un sistema di due nucleoni in uno stato legato con parità).
    \end{itemize}
    Pertanto il momento angolare iniziale è $j=1$. 
    \item Parità iniziale:
    \begin{itemize}
        \item La parità del sistema $\pi^-+d$ è data dalla parità del prodotto tra il $\pi^-$ e il deuterone:
        \begin{equation*}
        P\_{ingresso}=P_{\pi^-}P_d=(-1)\cdot (+1)= -1
        \end{equation*}
    \end{itemize}
\end{enumerate}
Canale di uscita: $n+n$
\begin{enumerate}
    \item Momento angolare totale $j$ finale:
    \begin{itemize}
        \item Gli stati possibili dei neutroni sono $s=0$ (singoletto) e $s=1$ (tripletto).
        \item Il momento angolare orbitale $l$ tra i due neutroni determinerà il valore di $j$ nel canale di uscita:
        \begin{equation*}
        j=l\pm s
        \end{equation*}
        \item Per i neutroni, che sono fermioni, il sistema complessivo deve essere antisimmetrico. Questo implica che se $s=0$ (singoletto), $l$ deve essere pari; se $s=1$ (tripletto), $l$ deve essere dispari.
    \end{itemize} 
    \item Parità finale (canale di uscita):
    \begin{itemize}
        \item La parità del sistema $n+n$ è data da:
        \begin{equation*}
        P\_{uscita}=(-1)^l
        \end{equation*}
        visto che i neutroni hanno parità $+1$.
    \end{itemize}
\end{enumerate}
\comment{Lo stato iniziale ha $l=0$ (preparato sperimentale) con $s_\pi=0$ e $s_d=1$ allora $j=1$. Dunque anche lo stato finale dovrà avere $j=1$. 
    La parità dello stato finale è data da 
    \begin{equation*}
    K=\underbrace{(-1)^{s+1}}_{\text{spin}}\underbrace{(-1)^l}_{\text{orbitale}}=(-1)^{l+s+1}
    \end{equation*}
    poiché sono fermioni}%fine commento
Allora per conservazione di momento angolare e parità, dovrò avere per gli stati finali 
\begin{gather*}
j\_i=j\_f\implies 1=l+s\\
P\_{ingresso}=P\_{uscita}\implies -1=(-1)^l\implies l \text{ dispari}
\end{gather*}
Visto che $s=0,1$ e $l$ deve essere dispari, allora $l=1$ e $s=1$ è l'unica combinazione che soddisfa entrambe le condizioni. Quindi il canale di uscita $n+n$ ha $l=1$ e $s=1$ (tripletto).
\subsubsection{Parità del pione neutro}
Il pione neutro decade con BR $=99\%$ in $\gamma+\gamma$. Questo decadimento è un processo elettromagnetico, quindi la parità si conserva.
