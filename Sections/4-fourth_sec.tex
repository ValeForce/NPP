pippo
\subsection{Simmetrie e leggi di conservazione}
In generale una legge fisica è simmetrica rispetto ad una trasformazione quando la forma della legge è invariante per questa trasformazione. Sia in meccanica classica che quantistica. 
\begin{itemize}
    \item In QM se l'operatore non dipende esplicitamente dal tempo allora commuta con la hamiltoniana del sistema. In generale (non sempre?) i numeri quantici conservati sono associati ad operatori che commutano con l'hamiltoniana.
    \item Le simmetrie si dividono in continue e discrete. Vediamo ad esempio la traslazione spaziale (continua)
    \begin{equation*}
        \psi(r+\delta r)=\psi(r)+\delta r \dv{\psi}{r}=\qty(1+\delta r \pdv{}{r})\psi(r)
    \end{equation*}
    L'operatore che descrive una traslazione finita è l'impulso:
    \begin{equation*}
        D=1+\frac i\hbar p\delta r\implies D=\lim_n\qty(1+\frac {ip\Delta r}{n\hbar})^n=e^{\frac i\hbar p\Delta r}
    \end{equation*}
    Chiamiamo $p$ generatore dell'operatore $D$ di traslazione spaziale. Se l'hamiltoniana è invariante per traslazioni, allora commuta con $D$ e dunque anche con $p$. Questo lo si può esprimere in tre modi equivalenti:
    \begin{enumerate}
        \item L'impulso si conserva in un sistema isolato.
        \item L'hamiltoniana è invariante per traslazioni spaziali.
        \item L'operatore impulso commuta con l'hamiltoniana.
    \end{enumerate}
    \item A simmetrie continue associamo numeri quantici additivi, a simmetrie discrete sono associati numeri quantici moltiplicativi.
\end{itemize}
\subsubsection{Parità}
La trasformazione di parità è l'inversione delle coordiante spaziali. 
\begin{equation*}
    P\psi(\vec r)=\psi(-\vec r)
\end{equation*}
Chiaramente se applico due volte l'operatore ottengo la funzione iniziale:
\begin{equation*}
P^2\psi(\vec r)=\psi(\vec r)\implies P^2=1 \text{ (unitario) }\implies \lambda=\pm 1
\end{equation*}
Vediamo degli esempi:
\begin{itemize}
    \item Consideriamo le semplici funzioni trigonometriche 
    \begin{gather*}
        \psi(x)=\cos x \overset{P}{\longrightarrow}\cos(-x)=\cos(x)=\psi(x)\text{ Pari} \\
        \psi(x)=\sin x \overset{P}{\longrightarrow}\sin(-x)=\sin(x)=-\psi(x)\text{ Dispari} 
    \end{gather*}
    In generale una combinazione lineare $\psi(x)=\cos x+\sin x$ non è detto che sia simmetrica per parità.
    \item Un altro esempio può essere la funzione d'onda di un elettrone in un atomo di idrogeno. La simmetria della funzione d'onda ha la stessa parità di $l$. Infatti
    \begin{equation*}
        \psi(r,\vartheta,\varphi)=\chi(r)\sqrt{\frac{(2l+1)(l-m)!}{4\pi(l+m)!}}P_m^l(\cos\vartheta)e^{im\varphi}
    \end{equation*}
    Fare una trasformazione di parità vuol dire 
    \begin{equation*}
    \vec r\to-\vec r\implies 
    \begin{cases}
        \vartheta\to\pi-\vartheta\\
        \varphi\to\pi+\varphi
    \end{cases}\implies 
    \begin{cases}
    e^{im\varphi}\to e^{im(\pi+\varphi)}=(-1)^me^{im\varphi}\\
    P_m^l(\cos\vartheta)\to (-1)^{l+m}P_m^l(\cos\vartheta)
    \end{cases}\implies 
    \end{equation*}
    \begin{equation*}
    \implies Y_l^m(\vartheta,\varphi)\to (-1)^{l+2m}Y_l^m(\vartheta,\varphi)=(-1)^lY_l^m(\vartheta,\varphi)
    \end{equation*}
    Questo risultato vale in generale per le armoniche sferiche, che quindi hanno parità data da $l$. Nelle transizioni di dipolo elettrico la regola di selezione è $\Delta l=\pm 1$, quindi la parità atomica cambia. Però nei processi elettromagnetici la parità si conserva, quindi la parità della radiazione emessa deve essere negativa per compensare la parità. 
    \item In questo caso il numero quantico è moltiplicativo e non si conserva soltanto nei decadimenti deboli. Inoltre è necessario che per convenzione assegniamo una parità intrinseca a ciascuna particella: a protoni e neutroni assegniamo parità positiva.
    \item Assegnare la parità intrinseca serve a distinguere particelle che interagiscono tra di loro (come cariche elettriche). Chiaramente il segno della parità intrinseca è scelto arbitrariamente, quello che conta è la parità relativa tra due particelle. Ad esempio particelle ed antiparticelle hanno parità opposta. Ad esempio nella reazione per la scoperta dell'antiprotone $p+p\to p+p+\bar p+p$ la parità totale nel canale di ingresso è uguale a quella in uscita (l'interazione forte la conserva). Questo discorso però funziona solo per i fermioni! Nel caso di fermioni, particella ed antiparticella hanno la stessa parità. 
    \item I vettori polari cambiano segno sotto trasformazione di parità e quelli assiali (pseudovettori) no.
    \begin{equation*}
    \begin{cases}
    \vec r\to-\vec r\\
    \vec p\to-\vec p\\
    \vec E\to-\vec E\\
    \end{cases}\text{ Polari}\qquad
    \begin{cases}
        \vec \sigma\to\vec \sigma\\
        \vec L\to\vec L\\
        \vec B\to\vec B\\
    \end{cases}\text{ Assiali}
    \end{equation*}
\end{itemize}
\subsubsection{Parità del pione carico}
Consideriamo il decadimento $\pi^-+d\to n+n$, che è un processo forte perché tutto si conserva.\\
Canale di ingresso: $\pi^-+d$
\begin{enumerate}
    \item Momento angolare totale $j$ iniziale:
    \begin{itemize}
        \item Supponiamo che lo stato sia preparato sperimentalmente con $l=0$.
        \item La particella $\pi^-$ ha spin $s=0$, quindi non contribuisce con il proprio spin. Inoltre ha parità negativa (è un cosiddetto mesone pseudoscalare).
        \item Il deuterone (d) ha $s=1$ e parità positiva (essendo un sistema di due nucleoni in uno stato legato con parità).
    \end{itemize}
    Pertanto il momento angolare iniziale è $j=1$. 
    \item Parità iniziale:
    \begin{itemize}
        \item La parità del sistema $\pi^-+d$ è data dalla parità del prodotto tra il $\pi^-$ e il deuterone:
        \begin{equation*}
        P\_{ingresso}=P_{\pi^-}P_d=(-1)\cdot (+1)= -1
        \end{equation*}
    \end{itemize}
\end{enumerate}
Canale di uscita: $n+n$
\begin{enumerate}
    \item Momento angolare totale $j$ finale:
    \begin{itemize}
        \item Gli stati possibili dei neutroni sono $s=0$ (singoletto) e $s=1$ (tripletto).
        \item Il momento angolare orbitale $l$ tra i due neutroni determinerà il valore di $j$ nel canale di uscita:
        \begin{equation*}
        j=l\pm s
        \end{equation*}
        \item Per i neutroni, che sono fermioni, il sistema complessivo deve essere antisimmetrico. Questo implica che se $s=0$ (singoletto), $l$ deve essere pari; se $s=1$ (tripletto), $l$ deve essere dispari.
    \end{itemize} 
    \item Parità finale (canale di uscita):
    \begin{itemize}
        \item La parità del sistema $n+n$ è data da:
        \begin{equation*}
        P\_{uscita}=(-1)^l
        \end{equation*}
        visto che i neutroni hanno parità $+1$.
    \end{itemize}
\end{enumerate}
\comment{Lo stato iniziale ha $l=0$ (preparato sperimentale) con $s_\pi=0$ e $s_d=1$ allora $j=1$. Dunque anche lo stato finale dovrà avere $j=1$. 
    La parità dello stato finale è data da 
    \begin{equation*}
    K=\underbrace{(-1)^{s+1}}_{\text{spin}}\underbrace{(-1)^l}_{\text{orbitale}}=(-1)^{l+s+1}
    \end{equation*}
    poiché sono fermioni}%fine commento
Allora per conservazione di momento angolare e parità, dovrò avere per gli stati finali 
\begin{gather*}
j\_i=j\_f\implies 1=l+s\\
P\_{ingresso}=P\_{uscita}\implies -1=(-1)^l\implies l \text{ dispari}
\end{gather*}
Visto che $s=0,1$ e $l$ deve essere dispari, allora $l=1$ e $s=1$ è l'unica combinazione che soddisfa entrambe le condizioni. Quindi il canale di uscita $n+n$ ha $l=1$ e $s=1$ (tripletto).
\subsubsection{Parità del pione neutro}
Il pione neutro decade con BR $=99\%$ in $\gamma+\gamma$. Questo decadimento è un processo elettromagnetico, quindi la parità si conserva. 
\begin{itemize}
\item Siano $\vec k$ e $-\vec k$ gli impulsi dei due fotoni e $\varepsilon_1$ e $\varepsilon_2$ i vettori polarizzazione.
\item Visto che i fotoni sono bosoni, la funzione d'onda totale deve essere simmetrica per scambio di particelle.
\begin{gather*}
\psi\text{ pari: } \psi_1(2\gamma)=A(\vec\varepsilon_1\cdot\vec\varepsilon_2)\propto\cos\varphi\\
\psi\text{ dispari: } \psi_1(2\gamma)=B(\vec\varepsilon_1\times\vec\varepsilon_2)\cdot \vec k\propto\sin\varphi
\end{gather*}
dove $\varphi$ è l'angolo tra i piani di polarizzazione dei fotoni. La $\psi_1$ è scalare (parità positiva), la $\psi_2$ è pseudoscalare (parità negativa). 
\item Quindi per ora non sappiamo quale delle due funzioni d'onda è quella giusta. Abbiamo i due casi 
\begin{gather*}
P_{\pi^0}=+1\implies \abs{\psi}^2\propto\cos^2\varphi\\
P_{\pi^0}=-1\implies \abs{\psi}^2\propto\sin^2\varphi
\end{gather*}
Per misurare devo studiare il decadimento del $\pi^0\to\gamma\gamma$. Noi ovviamente per rivelare i fotoni andiamo a cercare le coppie elettrone-positrone. Dunque cerchiamo 
\begin{equation*}
\pi^0\to\gamma\gamma\to e^+e^-e^+e^-
\end{equation*}
detto decadimento \textit{doppio Dalitz} con BR $(3.14\pm0.30)\cdot10^{-5}$. Da un grafico (che metterò mai visto che non passa le slide) vedo distribuzione sperimentale che è in accordo con andamento con parità negativa (funzione d'onda va con $\sin^2\varphi$).
\end{itemize}
\subsubsection{Conservazione della parità}
Abbiamo detto che la parità è conservata nei processi forti ed elettromagnetici e non nei processi deboli. 
\begin{itemize}
    \item Un esempio è il neutrino. Noi li conosciamo solo sinistri (left-handed) e non destri. Infatti un neutrino come lo conosciamo noi ha elicità negativa, cioè impulso e spin antiparalleli. Se facciamo una trasformazione di parità solo l'impulso cambia segno e non lo spin, quindi si avrebbe un \textit{neutrino destro} che in realtà non esiste, o meglio non interagisce con la materia attraverso le forze che conosciamo presenti nel Modello Standard, dunque non è rivelabile. 
    \item In realtà sperimentalmente è stata misurata una piccola violazione della parità in processi forti ed elettromagnetici. Ciò è dovuto al fatto che la hamiltoniana in realtà è composta da tre pezzi delle tre interazioni, quindi la parità non è conservata in generale perché c'è sempre un piccolo contributo di interazione debole. Ad esempio questo problema non sorge con l'elettrone, che sappiamo esistere sia destro che sinistro in quanto ha comunque la carica elettrica e quindi mal che vada elettromagneticamente lo riveliamo sempre. Il problema sussiste solo con il neutrino.
    \item Nella corrente carica, cioè scambio di $W^\pm$, la parità è violata al 100\%. Questo perché il $W$ interagisce sempre solo con particella sinistra o antiparticella destra (ignora le altre due tipologie, è razzista, non ci interagisce). Nel caso della corrente neutra, cioè scambio di $Z^0$, la situazione è più complessa. Infatti esso interagisce sia con particelle sinistre che destre, ma con pesi (accoppiamenti) diversi, di meno con particelle destre. Invece interazione forte ed elettromagnetica non distinguono elicità, cioè particelle destre o sinistre, e quindi la parità è conservata. 
\end{itemize}
\subsubsection{Particelle ed antiparticelle}
\begin{itemize}
    \item Al solito noi ci aspettiamo che esistono antiparticelle anche solo dalla relatività speciale in quanto ci sono soluzioni con energia negativa. 
    \item In meccanica quantistica rappresentiamo l'ampiezza di un flusso di particelle (e.g. elettroni) come una funzione d'onda piana
    \begin{equation*}
        \psi(x)=Ae^{\frac i\hbar(px-E t)}
    \end{equation*}
    questa espressione rappresenta anche particelle di energia $-E$ e impulso $-p$ che si muovono in direzione opposta nello spazio e nel tempo (anche in Klein-Gordon, per questo nei diagrammi di Feynman hanno direzione indietro nel tempo).
\end{itemize}
\subsubsection{Coniugazione di carica}
\begin{itemize}
    \item L'effetto dell'operatore coniugazione di carica $C$ è di invertire carica e il momento magnetico della particella.
    \item In fisica classica abbiamo che le leggi di Maxwell sono invarianti per esse, che sono 
    \begin{align*}
        q&\to-q\\
        \vec J&\to-\vec J\\
        \vec E&\to -\vec E\\
        \vec H&\to -\vec H
    \end{align*}
    \item In meccanica quantistica invece quando applichiamo $C$ abbiamo\\ 
    \begin{center}
    \begin{tabular}{>{\centering\arraybackslash}m{3cm} >{\centering\arraybackslash}m{3cm} >{\centering\arraybackslash}m{3cm}}
          & Protone & Antiprotone \\
        Carica & +e & -e \\
        N. barionico & 1 & -1 \\
        Momento magnetico & $\frac{e\hbar}{2mc}=2.79$ & -2.79 \\
        Spin & $\frac12\hbar$ & $-\frac12\hbar$ \\
    \end{tabular}
    \end{center}
    \item Consideriamo il neutrino sinistro. Se effettuiamo la trasformazione di parità otteniamo un neutrino destro (spin e impulso parallelo); se effettuiamo la trasformazione di coniugazione di carica otteniamo un antineutrino sinistro (spin e impulso antiparalleli). Entrambe particelle che \textit{non} esistono. Tuttavia se effettuiamo entrambe le trasformazioni, otteniamo un antineutrino destro, che è una particella che esiste. Quindi c'è buona simmetria $CP$ (o $PC$) che viene rispettata... c'è solo una piccolissima violazione. 
\end{itemize}
\subsubsection{Autostati dell'operatore $C$}
Solo i bosoni neutri che coincidono con la propria antiparticella possono essere autostati di $C$.
\begin{itemize}
\item Se applichiamo $C$ ad un pione carico non otteniamo un autostato:
\begin{equation*}
C\ket{\pi^+}=\ket{\pi^-}\neq\lambda\ket{\pi^+}
\end{equation*}
dunque i pioni carichi non sono autostati di $C$. 
\item Invece abbiamo, poichè applicando due volte $C$ torniamo allo stato di partenza $C^2\ket{\pi^0}=\ket{\pi^0}$,
\begin{equation*}
C\ket{\pi^0}=\lambda\ket{\pi^0}\implies C^2\ket{\pi^0}=\lambda^2\ket{\pi^0}=\ket{\pi^0}\implies \lambda=\pm 1
\end{equation*}
Per determinare se è positivo o negativo, ricordiamo che il decadimento è $\pi^0\to\gamma\gamma$ quindi $C(\pi^0)=+1$ perché l'interazione elettromagnetica è invariante per coniugazione di carica (la conserva). Lo si capisce anche classicamente, infatti cambiando la carica cambiano segno sia $\vec E$ sia $\vec B$. 
\item Dunque $C(\gamma)=-1$ e da ciò deduciamo che il decadimento $\pi^0\to3\gamma$ è proibito. D'altra parte troviamo sperimentalmente
\begin{equation*}
\frac{\text{BR}(\pi^0\to3\gamma)}{\text{BR}(\pi^0\to2\gamma)}<3.1\times10^{-8}
\end{equation*}
\end{itemize}
\subsubsection{Conservazione di $C$}
\begin{itemize}
\item La coniugazione di carica si conserva in interazioni forti ed elettromagnetiche. Allora nei processi forti mi aspetto sempre particelle ed antiparticelle per compensarsi.
\item Il mesone $\eta$ decade in vari modi ($J^P=0^-, m=550\MeV$).
\begin{enumerate}
    \item $\eta\to\gamma\gamma$ BR $=$ 39.4\%
    \item $\eta\to\pi^+\pi^-\pi^0$ BR $=$ 23.1\%
    \item $\eta\to\pi^+\pi^-\gamma$ BR $=$ 4.7\%
    \item $\eta\to\pi^0e^+e^-$ BR $<4\times10^{-5}$\%
\end{enumerate} 
poiché $\eta\to\gamma\gamma$ è il decadimento principale, si deve avere $C(\eta)=+1$. Dunque il decadimento $\eta\to\pi^0e^+e^-$ è proibito dalla conservazione di C. Ma questo è dovuto al fatto che $C(e^+e^-)=+1$. Vediamo perché.
\end{itemize}
\subsubsection{Decadimento del positronio}
\begin{itemize}
\item Il positronio è uno stato legato $e^+e^-$ che possiede livelli energetici simili all'atomo di idrogeno (con circa la metà dello spazio). La funzione d'onda la suddividiamo:
\begin{equation*}
\psi(e^+e^-)=\varphi(\text{spazio})\times\alpha(\text{spin})\times \chi(\text{carica})
\end{equation*}
\begin{enumerate}
    \item $\varphi(\text{spazio})$. Lo scambio di particelle è equivalente all'inversione spaziale, dunque questo termine introduce solo un fattore $(-1)^L$ dove $L$ è il momento angolare orbitale.
    \item $\alpha(\text{spin})$. Avendo due fermioni, si possono accoppiare in singoletto ($s=0$) o tripletto ($s=1$). Indicando gli stati con $\psi(s,s_z)$, abbiamo:
    \begin{equation*}
        \begin{cases}
            \alpha(1,1)=\psi_1(\frac12,\frac12)\psi_2(\frac12,\frac12)\\
            \alpha(1,0)=\frac1{\sqrt2}\qty(\psi_1(\frac12,\frac12)\psi_2(\frac12,-\frac12)+\psi_1(\frac12,-\frac12)\psi_2(\frac12,\frac12))\\
            \alpha(1,-1)=\psi_1(\frac12,-\frac12)\psi_2(\frac12,-\frac12)
        \end{cases}\quad
        \begin{array}{c}
            s=1\text{ Tripletto}\\
            \text{Simmetrico}
         \end{array} 
    \end{equation*}
    \begin{equation*}
        \alpha(0,0)=\frac1{\sqrt2}\qty(\psi_1\qty(\frac12,\frac12)\psi_2\qty(\frac12,-\frac12)-\psi_1\qty(\frac12,-\frac12)\psi_2\qty(\frac12,\frac12))\quad\begin{array}{c}
            s=0\text{ Singoletto}\\
            \text{Antisimmetrico}
         \end{array}
    \end{equation*}
    La simmetria della parte di spin è data da $(-1)^{s+1}$, dunque è pari se $s$ è dispari e viceversa.
    \item Per la parte carica invece consideriamo un fattore $C$.
\end{enumerate}
\item Dunque la simmetria totale della funzione d'onda per scambio di elettroni e positroni è 
\begin{equation*}
    K=(-1)^L(-1)^{s+1}C
\end{equation*}
siccome abbiamo un sistema di due fermioni, la simmetria totale deve essere antisimmetrica $K=-1$.
\item Allora con $L=0(\implies J=s)$ sperimentalmente posso avere due diversi decadimenti
\begin{equation*}
    e^+e^-\to2\gamma\qquad e^+e^-\to 3\gamma
\end{equation*}
e dunque possiamo avere (il vincolo è $K=-1=(-1)^{L+s+1}C$)
\begin{center}
    \begin{tabular}{>{\centering\arraybackslash}m{2.5cm} >{\centering\arraybackslash}m{2.5cm} >{\centering\arraybackslash}m{2.5cm}>{\centering\arraybackslash}m{2.5cm}>{\centering\arraybackslash}m{2.5cm}}
          & $J=S$ & $L$ & $C$ & $K$ \\
        $2\gamma$ & 0 & 0 &+1 &-1  \\
        $3\gamma$ & 1 & 0 & -1&-1 \\
    \end{tabular}
    \end{center}
perché sappiamo già che $C(n\gamma)=(-1)^n$.
\item Le larghezze di decadimento si possono confrontare teoricamente dalla QED e sperimentalmente. C'è accordo!
\begin{center}
    \begin{tabular}{>{\centering\arraybackslash}m{1cm} >{\centering\arraybackslash}m{4cm} >{\centering\arraybackslash}m{4cm}>{\centering\arraybackslash}m{4cm}}
          & $\Gamma$ & $\tau$ (teoria) & $\tau$ (esperimenti) \\
          $2\gamma$ & $\frac12mc^2\alpha^5$ & $1.252\times10^{-10}$ s &$\qty(1.252\pm0.017)\times 10^{-10} $s   \\
          $3\gamma$ & $\frac2{9\pi}(\pi^2-9)\alpha^6mc^2$ & $1.374\times10^{-7}$ s& $\qty(1.377\pm0.004)\times10^{-7}$ s \\
    \end{tabular}
    \end{center}
    Notiamo che entrambi sono possibili!
    \item Sommario:
    \begin{enumerate}
        \item Se ho un sistema di due mesoni, come pioni carichi:
        \begin{equation*}
            C\ket{\pi^+\pi^-;L}=(-1)^{L}\ket{\pi^+\pi^-;L}
        \end{equation*}
        perché in questo caso coniugazione di carica vuol dire scambiare la posizione, e spazialmente si ottiene sempre $(-1)^L$.
        \item Se ho un sistema fermione e antifermione, posso avere tripletto o singoletto e tutto il discorso fatto prima. Quindi
        \begin{equation*}
            C\ket{f\bar f;J,L,S}=\underbrace{(-1)^{L}}\_{spazio}\underbrace{(-1)^{S+1}}_{spin}(-1)\ket{f\bar f;J,L,S}
        \end{equation*}
        e l'ultimo $(-1)$ è dovuto al fatto che la coppia fermione antifermione ha parità intriseca negativa dal principio di Pauli. Per convenzione la particella ha $+1$ e l'antiparticella $-1$.
    \end{enumerate} 
\end{itemize}
\subsubsection{Gauge invarianza e conservazione di carica}
Ha detto due cose in croce. Il legame tra conservazione di carica e gauge invarianza.
\subsubsection{Teorema CPT}
\begin{itemize}
    \item Quando effettuiamo una inversione temporale $t\to-t$ le reazioni sono invarianti. Questo comporta che la sezione d'urto di una reazione è uguale a quella della reazione inversa, a meno di una piccola violazione. 
    \item Perché c'è questa violazione? In meccanica quantistica c'è un teorema che afferma che:\textit{ Tutte le interazioni sono invarianti per applicazione dei tre operatori $C$,$P$ e $T$ in qualunque ordine.} Dunque i processi che violano $CP$, violano anche $T$ così che si conservi tutto. Si hanno diverse conseguenze:
    \begin{enumerate}
    \item La massa della particella è uguale a quella della antiparticella. $\frac{m_{K^0}-m_{\bar K^0}}{m_{K^0}+m_{\bar K^0}}<10^{-19}$
    \item Il tempo di vita media è uguale tra particella ed antiparticella.
    \item Il momento magnetico è uguale ed opposto in segno tra particella ed antiparticella.
    \end{enumerate}
    \item La $CP$ è una buona simmetria, per le cose che facciamo è esatta. Nel 1964 studiando decadimenti di $K^0_L$ (long), che normalmente decadono in tre pioni ($CP=-1$), si è osservato che decadono anche in due pioni ($CP=+1$). Questa è una violazione di $CP$. Questa è l'unica fonte di spiegazione, nel Modello Standard, al fatto che nell'universo c'è asimmetria tra materia ed antimateria. 
    \item Dunque la violazione di $CP$ equivale ad una violazione di $T$ per il teorema $CPT$. I modi di osservare la violazione di $T$ sono due:
    \begin{enumerate}
        \item La polarizzazione trasversa $\vec\sigma\cdot(\vec p_1\times \vec p_2)$ nei decadimenti deboli come quello del muone.
        \item Il momento di dipolo elettrico $\vec\sigma\cdot\vec E$.
    \end{enumerate}
\end{itemize}
\subsubsection{Spin dei pioni carichi $\pi^\pm$\textbf{in realtà è legato a prima forse rendere subsection "time reversal"?}}
\begin{itemize}
\item Dalla inversione temporale possiamo avere informazione sullo spin dei pioni carichi.
\item È stato determinato guardando la reazione $p+p\rightleftharpoons\pi^++d$. La sezione d'urto praticamente ha la stessa formula della larghezza di decadimento, a meno di fattori dello spazio delle fasi. Abbiamo
\begin{gather*}
\sigma(pp\to\pi^+d)=\abs{M_{if}}^2\frac{\qty(2s_\pi+1)\qty(2s_d+1)}{v_iv_f}p_\pi^2\\
\sigma(\pi^+d\to pp)=\frac12\abs{M_{if}}^2\frac{\qty(2s_p+1)^2}{v_fv_i}p_p^2\qquad\text{Il fattore }\frac12\text{ viene dalla integrazione su metà angolo solido, poiché ci sono due fermioni identici nello stato finale}
\end{gather*}
\begin{equation*}
\implies \frac{\sigma(pp\to\pi^+d)}{\sigma(\pi^+d\to pp)}=2 \frac{\qty(2s_\pi+1)\qty(2s_d+1)}{\qty(2s_p+1)^2}\frac{p_\pi^2}{p_p^2}
\end{equation*}
Misurando le sezioni d'urto e conoscendo gli impulsi e spin di deuterio e protone, ricavo che $s_\pi=0$. Questo però è il pione carico.
\item Per lo spin del pione neutrone si guarda il decadimento $\pi^0\to\gamma\gamma$, che ci dice subito che lo spin deve essere intero e diverso da uno. I fotoni hanno $m=0,s=1$ e $s_z=\pm1$. Prendendo come asse di quantizzazione la direzione comune di propagazione dei fotoni nel sistema di riferimento del $\pi^0$, se \( S \) è lo spin totale dei due fotoni possiamo avere: \( S_z = 0 \) oppure \( S_z = 2 \). Se lo spin di $\pi^0$ è 1, allora \( S_z = 0 \). In questo caso l'ampiezza del sistema a due fotoni deve trasformarsi sotto rotazioni spaziali come il polinomio \( P^0_1(\cos \vartheta) \), che è dispari rispetto allo scambio dei due fotoni. Ma la funzione d'onda deve essere simmetrica rispetto allo scambio dei due bosoni identici, quindi lo spin di $\pi^0$ non può essere 1. In conclusione, \( s_\pi = 0 \) o \( s_\pi \geq 2 \). \href{https://chatgpt.com/c/6734e2f4-81d8-800b-8714-8b28ae3b1bc3}{ChatGPT}
\end{itemize}
\subsection{Piccolo excursus su numero barionico e leptonico}
pippo
\subsection{Isospin}
\begin{itemize}
\item Fu introdotta nel 1932 da Heisenberg notando la similitudine della massa tra protone e neutrone. Suppose che fossero stati di carica differente di una particella chiamata \text{nucleone}. Ad esempio osservando nuclei speculari come $^7Li$ e $^7Be$ si osserva che i livelli energetici hanno lo stesso pattern.
\item Al nucleone è associato il numero quantico di \textit{isospin}, conservato solo nelle interazioni forti. Il nucleone ha isospin $I=\frac12$, e le due proiezioni corrispondono a protone con $I_3=\frac12$ e neutrone con $I_3=-\frac12$.
\item Il nucleone ha un grado di libertà interno con due stati consentiti (il protone e il neutrone), che non sono distinti dalla forza nucleare. Scrivendo gli stati del nucleone come $\ket{I,I_3}$, si può scrivere la stessa cosa di prima per un sistema a due nucleoni con tripletto $I=1$ e singoletto $I=0$.
\item Vale l'importante Gell-Mann Nishijima.
\begin{equation*}
Q=I_3+\frac12 Y\qquad\text{con }Y=B\text{ (ipercarica)}
\end{equation*}
L'interazione elettromangetica rompe la simmetria di isospin, da cui si ha una differenza in massa tra protone e neutrone (o anche tra pioni carichi e neutri).
\item Ad esempio per i pioni che sono mesoni quindi $B=0$ abbiamo $Q=I_3$, dunque abbiamo un tripletto di isospin perché $Q=1,0,-1$.
\end{itemize}
\subsubsection{Isospin del deuterone}
\begin{itemize}
\item Sappiamo che il deuterone è uno stato legato $pn$ in $s$-wave cioè $l=0$ ed è in tripletto. Suddiviamo la funzione d'onda in parte spaziale, di spin e di isospin.
\begin{equation*}
\psi=\underbrace{\varphi(\text{spazio})}_{
\begin{subarray}
    ((-1)^L=+1\\
    (L=0)
\end{subarray}    
}\times\underbrace{\alpha(\text{spin})}_{
    \begin{subarray}
        ((-1)^{S+1}=+1\\
        (S=1)
    \end{subarray}    
    }\times\underbrace{\chi(\text{isospin})}_{(-1)^{I+1}}
\end{equation*}
La funzione d'onda per due fermioni identici (nucleoni) deve essere antisimmetrica. Questo implica che il deuterone ha spin nullo
\begin{equation*}
    (-1)^{I+1}=-1\implies I_d=0
\end{equation*}
\item Dunque il deuterone è uno stato singoletto anche per l'isospin. Infatti se consideriamo in generale due nucleoni, possiamo avere il tripletto simmetrico $I=1$ o il singoletto antisimmetrico $I=0$. Però se avessimo il tripletto, dovremmo osservare anche sistemi legati $pp$ e $nn$, che non esistono.
\end{itemize}
\subsubsection{Altri esempi di isospin}
\begin{itemize}
\item Consideriamo le due reazioni.
\begin{gather*}
p+p\to\pi^++d\\
p+n\to\pi^0+d
\end{gather*}
Poiché $I_\pi=1$ e $I_d=0$, allora gli stati finali hanno entrambi isospin pari ad 1. Gli stati iniziali invece sono:
\begin{gather*}
pp=\ket{1,1}\\
np=\frac1{\sqrt 2}\qty(\ket{1,0}-\ket{0,0})
\end{gather*}
La sezione d'urto è proporzionale all'ampiezza al quadrato
\begin{equation*}
\sigma\propto\abs{\text{ampiezza}}^2 \approx \sum_I \abs{\bra{I',I'_3}A\ket{I,I_3}}^2
\end{equation*}
e da conservazione di isospin (sono processi forti) abbiamo che $I=I'=1$ e $I_3=I'_3$. 
\item La reazione $np\to\pi^0d$ va come $\qty(\frac1{\sqrt2})^2$ rispetto alla reazione $pp\to\pi^+d$:
\begin{equation*}
\frac{\sigma(np\to\pi^0d)}{\sigma(pp\to\pi^+d)}=\frac12
\end{equation*}
ed è importante perché al solito misurando una, la più semplice da effettuare sperimentalmente, trovo l'altra.
\end{itemize}
\subsubsection{Isospin per sistema nucleone-nucleone}
\begin{itemize}
\item Consideriamo i seguenti processi
    \begin{gather*}
    \text{a) }p+p\to d+\pi^+\\
    \text{b) }p+n\to d+\pi^0\\
    \text{c) }n+n\to d+\pi^-
    \end{gather*}
\item Poichè il deuterone ha $I=0$, per gli stati finali abbiamo 
\begin{gather*}
    d+\pi^+=\ket{1,1}\\
    d+\pi^0=\ket{1,0}\\
    d+\pi^-=\ket{1,-1}
\end{gather*}
mentre per quelli iniziali
\begin{gather*}
    p+p=\ket{1,1}\\
    p+n=\frac1{\sqrt2}\qty(\ket{1,0}+\ket{0,0})\\
    n+n=\ket{1,-1}
\end{gather*}
\item Poiché l'isospin totale si deve conservare, solo gli stati con $I=1$ contribuiscono. Dunque per i tre processi le ampiezze di scattering sono nel rapporto:
\begin{equation*}
    1:\frac1{\sqrt2}:1\implies\text{per la }\sigma\,\, 2:1:2
\end{equation*}
\item I processi a) e b) sono stati misurati e, tenendo conto dell'interazione elettromagnetica, sono nel rapporto previsto.
\end{itemize}
\subsubsection{Isospin per sistema nucleone-mesone}
\begin{itemize}
    \item Poichè esistono tre pioni, gli assegniamo $I_\pi=1$, con carica data da $Q=I_3$ perché $B=0$. Dunque abbiamo
    \begin{equation*}
        \ket{\pi^+}=\ket{1,1}\quad\ket{\pi^0}=\ket{1,0}\quad\ket{\pi^-}=\ket{1,-1}\quad\ket{p}=\ket{\frac12,\frac12}\quad\ket{n}=\ket{\frac12,-\frac12}
    \end{equation*}
    e ricordiamo che le regole sono:
    \begin{equation*}
        \abs{I^1-I^2}\leq I\leq I^1+I^2\qquad I_3=I_3^1+I_3^2
    \end{equation*}
    \item Nel sistema $\pi$N l'isospin totale può essere $\frac12$ o $\frac32$, perché devo combinare isospin 1 e $\frac12$. Vediamo le diverse possibilità:
    \begin{gather*}
        \pi^+p\to\pi^+p\\
        \pi^-n\to\pi^-n\\
        \pi^-p\to\pi^-p\\
        \pi^-p\to\pi^0n\\
        \pi^+n\to\pi^+n\\
        \pi^+n\to\pi^0p
    \end{gather*}
    \item Le prime due sono degli stati puri di isospin $\frac32$. Questo perché $I_3=1+\frac12=\frac32$, dunque il caso $I=\frac12$ è proibito (cioè NON posso avere $\ket{\frac12,\frac32}$).
    \item Tutte le altre in generale sono combinazioni lineari dei due casi, con ampiezze date dai coefficienti di Clebsch-Gordan. Consideriamo ad esempio $\pi^+n\to\pi^0p$. Abbiamo 
    \begin{equation*}
        \ket{\pi^+n}=\ket{1,1}\times\ket{\frac12,-\frac12}=\alpha\ket{\frac32,\frac12}+\beta\ket{\frac12,\frac12}
    \end{equation*}
    perché in questo caso sono consentiti i due valori di isospin dato che $I_3=\frac12$. I valori dei coefficiente $\alpha$ e $\beta$ si ottengono dalla tabella dei coefficienti di Clebsch-Gordan (Guida a mano, in realtà basta leggere come al solito. Il termine che sceglie la tabella è $I^1\times I^2$; poi guardi le componenti di isospin delle due particelle e le cerchi a sinistra selezionando la riga; infine in alto consideri isospin e sua componente dello stato di cui vuoi trovare il coefficiente).
\end{itemize}
