Adesso vediamo le proprietà dei quark e dei gluoni da un altro punto di vista. 
\subsection{Esperimento di scattering}
\begin{itemize}
    \item Vogliamo sapere se il bersaglio è un semplice oggetto puntiforme, e se non lo è dobbiamo capire come sondare la sua struttura. 
    \item Per fare ciò ci serviamo di esperimenti di scattering. Potremmo fare scattering di Rutherford (1911) ma utilizzò particelle $\alpha$ e non riusci neanche a vedere la dimensione del nucleo (di oro). Infatti successivamente tra 1950-1960 Hofstadter arrivò alla struttura nucleare con elettroni in nuclei di $H/D/He$ e infine tra 1965-1980 (SLAC/CERN) con elettroni siamo arrivati alla materia adronica cioè i quark.
    \item Usiamo come sonda una particella elementare, normalmente un elettrone, perché oltre ad essere semplici da produrre, vogliamo essere in grado di scegliere noi la energia della sonda e soprattutto perché così siamo "sicuri" che il proiettile sia elementare e tutti gli effetti di non-elementarietà sono dovuti esclusivamente al bersaglio.
    \item Dunque in ordine:
    \begin{enumerate}
        \item Si sceglie una sonda (e.g. elettrone)
        \item Si studia lo scattering $e^-$-target 
        \item Si misura la sezione d'urto di $e^-T$, la distribuzione angolare di $e^-$ e si rivelano stati eccitati o lo stato finale del sistema adronico (scattering anelastico)
    \end{enumerate}
    \item Il modo di procedere è:
    \begin{enumerate}
        \item Si studia la cinematica. Con cinematica intendiamo le equazioni che seguono dalla conservazione di momento angolare e massa. Dopo aver imposto i vincoli cinematici si studia la \textit{dinamica}. 
        \item Calcoliamo la $\sigma(e^-T)$ per nuclei puntuali in elettrodinamica classica (formula di Rutherford).
        \item Si fa lo stesso per la meccanica quantistica con elettroni ($s=\frac12$) e nuclei puntuali (formula di Mott)
        \item Si rivelano \textit{deviazioni} da questi modelli. Così derivo informazioni sulla struttura nucleare
        \item Formulo una nuova teoria, poi vado a distanze ancora più piccole ($Q^2$ maggiore) e vedo se la nuova teoria regge o ci sono altre deviazioni e ripeto, potenzialmente fino all'infinito.
    \end{enumerate}
\end{itemize}
\begin{figure}[H]
    \centering
    \includegraphics[width=0.7\textwidth]{immagini/fig_treasure_map_scattering}
    \caption{Mappa del tesoro dello scattering.}
    %\label{}
\end{figure}
\subsection{Modello a gas di Fermi}
\begin{itemize}
    \item I nuclei sono stati legati di protoni e neutroni. Il gas di Fermi è un semplice modello.
    \item I protoni e i neutroni sono identici a meno della carica: sono delle sfere con una certa massa; sono fermioni; sono legati all'interno del nucleo, altrimenti sono liberi di muoversi. 
    \item Non consideriamo interazione elettromagnetica, solo nucleare dunque $N=Z=\frac A2$ ed impulso ed energia di Fermi sono uguali per protoni e neutroni (in approssimazione migliore sono differenti gli impulsi). 
    \item Per il principio di indeterminazione ogni $p$ ed $n$ occupano un volume $V=(2\pi\hbar)3$ nello spazio delle fasi.
    \item Dunque abbiamo una buca ben definita identica per protoni e neutroni, e viene rispettata la statistica di Fermi dunque due $p/n$ per livello di energia (con spin opposto).
    \item Da queste approssimazioni possiamo fare dei calcoli semplici:
    \begin{gather*}
    n_n^\uparrow=n_n^\downarrow=n_p^\uparrow=n_p^\downarrow=\frac N2=\frac Z2=\frac A4=\\
    =\frac{[V\_{space}V\_{imp}]\_{TOT}}{[V\_{space}V\_{imp}]\_{ciascuna particella}}=\frac{\frac43\pi r_0^3A\times\frac43\pi p_F^3}{(2\pi\hbar)^3}=\frac{2Ar_0^3p_F^3}{9\pi^2\hbar^3}\implies\\
    \implies N=Z=\frac A2=\frac{4Ar_0^3p_F^3}{9\pi^2\hbar^3}\implies p_F=\frac\hbar{r_0}\sqrt[3]{\frac{9\pi}8}\underset{r_0\approx1.2\,\text{fm}}{\implies}
    \begin{cases}
    p_F\approx250\,\MeV\\
    E_F^{\text{kin}}=\frac{p_F^2}{2m}\approx33\,\MeV
    \end{cases}
    \end{gather*}
    dove il valore di $r_0$ viene da fit del fattore di forma (vedi dopo) e per $E^{\text{kin}}_F$ si è considerata approssimazione non relativistica. 
    \item In conclusione: $V\_{space}\approx\frac43\pi r_0^3A\implies r\_{nucl}\propto A^{1/3}$, impulso ed energia di Fermi non dipendono da $A$ e ad un largo $p_F$ corrisponde bassa energia cinetica ed infine quando $p/n$ sono colpiti da una sonda ($e^\pm/\nu$), se l'energia della sonda è superiore a 30 MeV, si ignora il moto di Fermi.  
\end{itemize}
\subsection{Scattering di Rutherford}
\begin{itemize}
    \item L'esperimento fu eseguito a Manchester nel 1908-1913. Consiste in particelle $\alpha$ contro un bersaglio di oro $Au$.
    \item Propose un modello diverso da quello di J.J. Thompson.
    \item Fu il primo esperimento di scattering di particelle, infatti è la nascita della fisica nucleare.
    \item Dunque si mandarono queste particelle $\alpha$ su un foglio di oro con energia cinetica di qualche MeV. Qualche volta l'angolo di scattering era maggiore di 90$^\circ$ che nella pratica è un evento molto raro ma impossibile se la materia fosse effettivamente omogenea come nel modello di Thompson. L'unica possibile spiegazione è che la materia in realtà è concentrata in piccoli corpi pesanti, i nuclei. Dunque la materia è essenzialmente vuota.
    \item Come modelliamo lo scattering? Rutherford provo con una diffusione a due corpi. Chiaramente solo contributo elettrostatico non relativostico e senza meccanica quantistica. Fu un successo.
    \item Un punto chiave è che il nucleo è abbastanza piccolo da essere visto puntiforme (e con tutta la sua carica) dalla particella $\alpha$ (teorema di Gauss, primo capitolo).
    \item Tuttavia la materia è neutra, ma il fatto è che gli elettroni sono molto leggeri e non possono fermare o deflettere le particelle $\alpha$ perché $m_\alpha\approx8000m_e$.
    \item Vediamo i conti per la sezione d'urto\footnote{Stavo per trascriverli ma non sono io quello col nickname \textit{Il flagellante}}.
    \begin{figure}[H]
        \centering
        \includegraphics[width=\textwidth]{immagini/fig_rutherford_math.png}
        \caption{$\beta$ è l'angolo tra la particella e l'asse di simmetria (che passa per il punto di minimo approccio).}
        %\label{}
    \end{figure}
    \item Se la forza è attrattiva la sezione d'urto è uguale perché cambia solo
        \begin{equation*}
    \vec F\to -\vec F\implies\vartheta\to-\vartheta
    \end{equation*}
    \item Adesso troviamo espressione esplicita per la $r\_{min}$. Cominciamo una particella con $b=0\implies \vartheta=\pi$. Avremo $d_0=r\_{min}(b=0)$ dato da
    \begin{equation*}
    \frac12mv^2=\frac{zZe^2}{4\pi\varepsilon_0d_0}\implies d_0=\frac{zZe^2}{2\pi\varepsilon_0mv^2}\implies \tan\qty(\frac\vartheta2)=\frac{d_0}{2b}
    \end{equation*}
    Per trovare $d$ ($b\neq0$) usiamo conservazione di $\vec L$ ed $E$ (dove $v_0$ è la velocità in $d$)
    \begin{equation*}
        \begin{cases}
        mbv=mdv_0\to \frac{v_0}v=\frac bd\\
        \frac12mv^2=\frac12mv_0^2+\frac{zZe^2}{4\pi\varepsilon_0d}=\frac12mv_0^2+\frac12mv^2\frac{d_0}d
        \end{cases}\implies\qty(\frac bd)^2=\qty(\frac{v_0}v)^2=1-\frac{d_0}d\implies 
    \end{equation*}
    \begin{equation*}
        \implies d^2-dd_0-b^2=0\implies d=\dots=\frac{d_0}2\qty(1+\frac1{\sin(\frac\vartheta2)})\implies \dv{\sigma}{\Omega}= 
    \end{equation*}
\end{itemize}