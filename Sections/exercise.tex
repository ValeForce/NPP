
Esercizi di cinematica relativistica.
\begin{itemize}
    \item $m_\mu=106$ MeV, $\tau_\mu=2.2\cdot10^{-6}$s, $p_\mu=10$ GeV, $l=10km$. Calcolare la probabilità di avere il muone sulla superficie nel riferimento del laboratorio e in quello solidale al muone. Si parte da $\dv{\probP}{t}=\frac1\tau e^{-t/\tau}\implies \frac1{\tau\beta c}e^{-x/\beta c\tau}$, con $t=\frac x{\beta c}$. Allora si ottiene  (ricordiamo $\beta\gamma c=\frac p m$)
    \begin{equation*}
    \probP=\int_0^{d/\gamma}\frac1{\tau\beta c}e^{-x/\beta c \tau}\dd{x}=\dots =0.85
    \end{equation*}
    85\% è la probabilità che il muone arrivi sulla superficie prima di decadere nel riferimento solidale. Classicamente non è possibile. Invece nel riferimento di laboratorio si fa lo stesso calcolo ma con tempo dilatato e lunghezza non contratta. Si ottiene esattamente lo stesso valore, questo è dovuto al fatto che la probabilità è indipendente dal sistema di riferimento.
    \item Stesso principio dello scorso esercizio, si usa in fisica degli acceleratori quando serve trasportare per lunghe distanze i fasci prima di collidere contro il bersaglio. $m_\pi=140$ MeV, $\tau_\pi=2.56\cdot10^{-8}$s, $p_\pi=200$ GeV. Il limite galileiano $c\tau\approx 8$m, poco rispetto ai 300 m del nostro problema. La $\probP\_{classica}=e^{-\frac{ct}{c\tau}}\sim 10^{-17}$ bassissima. Invece per quella relativistica ricaviamo $\gamma=\frac E {m}=1429$. Ricaviamo il $\tau\_{lab}=\gamma \tau =3.71\cdot 10^{-5}$s e il tempo da attraversare $t=l=10^{-6}$s. Allora la probabilità vale $\probP\_{rel}=e^{-t/\tau\_{lab}}=0.97$. Se volessi la distanza $d=\beta \gamma \tau=11$km.
\end{itemize}
Esercizi di pippo