\subsection{Particelle elementari ed interazione}
Una particella si dice elementare se non possiede una struttura interna. 
\begin{itemize}
    \item Una particella elementare è tale in base al tempo in cui ci troviamo: cambia in base alle nostre conoscenze. Una volta l'atomo era considerato elementare, adesso sappiamo che c'è un nucleo, che è composto a sua volta da nucleoni che è composto a sua volta da quark. Questo è ciò a cui siamo arrivati oggi, non possiamo essere sicuri che i quark siano elementari e quindi che non abbiano una struttura interna.
    \item Con energie maggiori, siamo in grado di migliorare la nostra risoluzione e poter sondare strutture più piccole, cioè distanze piccole. Questo viene dalla meccanica quantistica e la relazione di De Broglie. 
    \item Un sistema come il nucleo ha dei livelli e questo è dovuto intrinsecamente al fatto che c'è una struttura interna e i nucleoni possono ri-organizzarsisu livelli diverse.
    \item Oggi con LHC arriviamo a 14 TeV, e così siamo arrivati ai quark. Magari migliorando la risoluzione, cioè aumentando l'energia, scopriamo una struttura interna ai quark.
    \item L'interazione tra particelle avviene per scambio di particelle mediatrici (non materiali). Queste particelle mediatrici sono dette bosoni e hanno spin intero (uno).
\end{itemize}
Le scale di energia sono:
\begin{itemize}
    \item Per cristalli e molecole si parla di cm a cui corrispondono decine di eV.
    \item Per atomi si parla di $10^{-10}$ m.
    \item Per i nuclei si parla di $10^{-15}$ m a cui corrispondono fino a centinaia di MeV.
    \item Per le particelle elementari fino ad 1 TeV.\textbf{VEDERE SLIDE DIMENSIONI DEI QUARK, SE C'È}
\end{itemize}
Parliamo del Modello Standard. Sappiamo che ci sono 3 famiglie o generazioni di particelle elementari della materia, che si suddividono in quark e leptoni (sono tutti fermioni). Ricordiamo inoltre che il Modello Standard è basato sul fatto che la massa del neutrino è nulla.
\begin{itemize}
    \item Le famiglie di leptoni sono 
        \begin{equation*}
            L: \binom{\nu_e}{e^-}, \binom{\nu_\mu}{\mu^-}, \binom{\nu_\tau}{\tau^-}
        \end{equation*}
    \item Mentre di quark sono
        \begin{equation*}
            Q: \binom{u}{d}, \binom{c}{s}, \binom{t}{b}
        \end{equation*}
    \item Sono raggruppati in doppietti perché è sottointesa una simmetria, dovuta alla interazione debole. Si può notare che la differenza di carica tra particella alta e bassa è sempre di 1. Inoltre la particella superiore ha sempre carica maggiore di quella inferiore. Questi fatti sono dovuti al atto che si passa da una all'altra tramite interazione debole con scambio di bosoni W$^\pm$, che è quindi "l'accoppiatore" di queste particelle di ciascuna famiglia. La massa delle famiglie va ad aumentare con il numero di famiglia, che comunque non è un parametro rilevante nella loro interazione.
    \item I leptoni sono sempre soggetti a forza debole, invece sono soggetti a forza elettromagnetica solo se carichi. Invece i quark sono sempre soggetti a forza forte, ed a forza elettromagnetica.
    \item I mediatori dell'interazione elettromagnetica sono i fotoni, quelli della interazione forte sono i gluoni (otto), quelli della interazione debole sono i bosoni W$^\pm$ e Z.
    \item La gravità chiaramente agisce su ogni particella in quanto sono dotate di masse. Sul gravitone non si hanno evidenze sperimentali. Ci piacerebbe che esistesse così da poter descrivere la gravità al pari delle altre tre interazioni. Ad ogni modo la sua intensità è 39 ordini di grandezza più piccola rispetto alla interazione forte quindi è molto difficile da osservare.
\end{itemize}
\subsection{Evoluzione storica}
Vediamo come si è arrivati al Modello Standard.
\begin{itemize}
    \item Inizialmente, tra il 1700 e il 1800 da studi di reazioni chimiche si ottennero le varie leggi di Dalton, Boyle etc. Daltone giunse alla conclusione che l'atomo fosse la particella costituente della materia e che fosse indistruttibile e indivisibile. In generale la materia è fatta da atomi diversi. Avogadro aggiunse l'esistenza delle molecole, aggregazioni di atomi. 
    \item C'erano 92 elementi la cui massa si poteva sempre esprimere come multiplo del primo elemento cioè l'idrogeno. Questo ci fa pensare che dietro si nasconda una simmetria, ossia c'è qualcosa che si ripete.
    \item Si può stimare il raggio atomico conoscendo densità \textbf{rivedi slide} e assumendo volume di una sfera. Otteniamo $\qty(\frac{3}{4\pi n}f)^\frac13$ con $n$ numero di atomi per unità di volume e $f$ fattore che tiene conto dell'impacchettamento, cioè quanto sono vicini o lontane le particelle nell'atomo. Si ottiene una stima sui $10^{-10}$ m.
\end{itemize}
Parliamo della tavola periodica.
\begin{itemize}
    \item Essa non può rappresentare le particelle elementari innanzitutto per una questione filosofica: non possono essere così tante le particelle elementari. In realtà niente lo vieta, ma semplicemente non ce lo aspettiamo.
    \item Un fattore più importante è la regolarità delle proprietà chimico-fisiche degli elementi in essa. Questo nasconde la presenza di struttura interna.
    \item Ad ogni modo ha molte informazioni. È difficile individuare questo tipo di simmetrie, però sappiamo che qualcosa che si ripete c'è.
    \item Dunque inizialmente la particella elementare era l'atomo di idrogeno, con tutti gli atomi proporzionali ad esso.
    \item Successivamente Thomson scoprì l'elettrone di massa 2000 volte minore rispetto all'idrogeno. Questo destabilizza la nostra conoscenza, perché l'atomo è neutrone ed è stato scoperta qualcosa di negativo al suo interno. Quindi qualcosa doveva compensare la carica negativa dell'elettrone all'interno dell'atomo. In effetti già questa era la prova che l'atomo non fosse elementare.
    \item Rutherford quindi testò il modello a panettone di Thomson e scoprì che l'atomo è composto da un nucleo e da elettroni che orbitano attorno ad esso. Questo è il modello planetario. Ciò era dovuto al fatto che mandando un fascio di $\alpha$ contro un foglio d'oro si osservava che la maggior parte delle particelle passava dritto, ma alcune venivano deviate di molto, alcune addirittura backscatterate. Questo è dovuto al fatto che l'atomo è composto da un nucleo molto piccolo rispetto al volume dell'atomo, mentre se fosse vero il modello di Thomson le particelle si sarebbero dovute deviare di poco. Questa scoperta fu possibile solo alla scoperta della radioattività naturale, infatti per generare il fascio di $\alpha$ si usò il polonio che è radioattivo.
    \item Approfondiamo questo aspetto. Se mandiamo un fascio di $\alpha$ contro un foglio d'oro, se la carica positiva è diffusa su tutto l'atomo allora in base al parametro d'impatto del fascio, esso vedrà una carica ridotta (cioè non tutta) secondo il teorema di Gauss in base al parametro d'impatto. Si ha che la carica dentro e fuori si compensano e quindi non si dovrebbe avere una grande deflessione, mentre si osservò l'opposto.
\end{itemize}
\subsection{Sonde sperimentali}
\begin{itemize}
    \item Per la scelta di una sonda l'elemento chiave è la risoluzione. Il motivo è legato all'ottica. Quando mandiamo onde contro delle fenditure, si devono confrontare l'elemento geometrico (in questo caso l'apertura della fenditura) e la lunghezza d'onda dell'onda incidente. Questo è ciò che dobbiamo fare anche in meccanica quantistica. Minore sono le lunghezze d'onda, maggiore sarà la risoluzione e, ricordando la relazione di De Broglie, maggiore deve essere l'energia. Questa è la base della fisica degli acceleratori. C'è dunque un legame tra la lunghezza d'onda incidente e un fattore geometrico dell'oggetto da osservare.
    \item Grazie effetto fotoelettrico e relazione di De Broglie c'è completo legame tra onde e particelle. 
    \item Rutherford infatti riuscì nel suo esperimento perché la lunghezza d'onda delle particelle $\alpha$ era vicina alle dimensioni che oggi sappiamo essere del nucleo, ossia $10^{-15}$ m. Quindi aveva risoluzione esatta. Impiegò $v_\alpha=0.05$ c (vale espressione di impulso classica). $\lambda_\alpha=\frac\hbar{m_\alpha v_\alpha}\approx 10^{-15}$ m. Se di 1 MeV avesse usato energie del keV non avrebbe visto nulla. 
    \item In generale quindi se ho lunghezze d'onda maggiori del raggio nucleare, le particelle incidenti non riescono a vedere il nucleo e interagiscono solo con nube elettronica, portando a debole scattering (cioè piccole deflessioni); se le due dimensioni sono comparabili, si osservano forte deflessioni come in Rutherford e si risolve la struttura nucleare; se la lunghezza d'onda è inferiore al raggio nucleare, non solo esploriamo il nucleo ma anche i costituenti dei nucleoni. Pertanto, la lunghezza d'onda gioca un ruolo chiave nel determinare cosa si possa "vedere" e quali fenomeni si osservano a livello nucleare o subnucleare.
    \item Tipicamente invece di parlare di lunghezza d'onda si parla di quadrimpulso trasferito $q^2$. Esso è collegato al potere risolutivo (se $q^2$ è grande, la risoluzione è grande).
\end{itemize}
\subsection{Scala di energia}
\begin{itemize}
    \item In fisica delle particelle l'energia la si dà in multipli di eV. Nel LHC un protone ha energia di 6.5 TeV. Questi corrispondono a $10^{-6}$ J, che in scala microscopica è enorme, mentre in scala macroscopica è una energia insignificante. Quindi è rilevante sapere il sistema di cui si sta parlando, oltre all'ordine di grandezza.
    \item Si usano unità naturali ecc. \textbf{Mettere la tabella dalle slide}.
    \item Un altro punto importante è la analisi dimensionale. Usiamo unità naturali. Al solito tempo e spazio sono omogenei e sono inversi all'energia. \textbf{Inserire altra tabella}.
\end{itemize}
\subsection{Relazioni importanti}
Al solito valgono le formule relativistiche (uso unità naturali).
\begin{itemize}
    \item Sappiamo che $E=m=\frac{m_0}{\sqrt{1-v^2}}$. Se $v\ll 1$ (cioè $v\ll c$) allora $E\approx m_0+\frac12 m_0v^2$ sviluppando in serie.
    \item Inoltre $p=mv=\frac{m_0v}{\sqrt{1-v^2}}$ e $E^2=p^2+m^2$. Per il fotone $E=p$.
    \item Facciamo un esempio numerico. Supponiamo di avere un elettrone ($m=0.511 MeV$ con $v=0.99$. Quindi $\gamma = 7.089$. Allora $E=3.62 MeV$ e $p=3.58 MeV$, cioè sono molto vicini! Questo è dovuto al fatto che la massa è piccola rispetto all'energia. Questa approssimazione la facciamo \textbf{sempre}, cioè la massa la poniamo a zero perché trascurabile rispetto all'impulso che ha la particella. 
    \item Se invece $v=0.999$, l'energia e l'impulso erano ancora più vicini.
\end{itemize}
\subsection{Evoluzione post-Rutherford}
\begin{itemize}
    \item Allora l'atomo è neutro. Si suppose inizialmente che avesse semplicemente tanti protoni quanti elettroni e che il responsabile della massa dell'atomo fosse il nucleo.
    \item Ma se fossi così, la massa non tornerebbe con le misure sperimentali. Deve esistere dunque altro. Deve essere neutro perché la carica è già a posto, e deve avere massa simile ai protoni perché la massa misurata era circa il doppio di quella prevista considerando solo protoni.
    \item Con l'idea del neutrone nasce anche l'idea di una nuova interazione, quella forte che si aggiunge a quella gravitazionale e quella elettromagnetica. Infatti fino ad ora quella elettromagnetica era responsabile di tutto, ma non può invece spiegare come mai i neutroni sono legati.
    \item Il problema era evidente considerando l'anomalia del $^{14}$N. \textbf{GUARDA QUADERNO RIZZO} Ma se l'atomo è composto da fermioni, allora nel sistema con 14 protoni e 7 elettroni, avendo 21 particelle avrò per forza spin semintero perché le accoppio a due a due, mentre si misura spin pari a 1. 
    \item Se invece considero il neutrone, allora ho 7 protoni, 7 neutroni e 7 elettroni, quindi ho 28 particelle e quindi spin intero.
\end{itemize}
\subsection{Scoperta del neutrone di Chadwick}
Non approfondiremo l'apparato sperimentale più di tanto, ma ci concentreremo su altre questioni.
\begin{itemize}
    \item L'esperimento per la scoperta del neutrone fu fatto già prima di Chadwick (1932) ma fu mal interpretato. Cerchiamo di capire perché.
    \item La reazione coinvolta è $ \alpha + ^9_4Be \rightarrow ^{12}_6C + n$, che sappiamo essere quella corretta (interpretabile come scattering $\gamma + n \rightarrow \gamma + n$). Sul canale d'ingresso non abbiamo dubbi, perché le particelle $\alpha$ le forniamo noi dalla sorgente di polonio e il target di berilio lo abbiamo scelto noi. I problemi sorgono sul canale di uscita. 
    \item Nella interpretazione scorretta della reazione, senza considerare il neutrone, si supponeva che la reazione fosse $ \alpha + ^9_4Be \rightarrow ^{13}_6C + \gamma$ (oppure $\gamma + p \rightarrow \gamma + p$). Infatti ai tempi si sapeva solo che ci fosse radiazione neutra molto penetrante. 
    \item \textit{Quale è la differenza?} Se fosse davvero un $\gamma$, avrebbe energia molto elevata, che in realtà non ho a disposizione (50 MeV)
\end{itemize}
Vediamo l'apparato sperimentale. \textbf{METTI IMMAGINE SLIDE}
\begin{itemize}
    \item Se mandiamo un fascio di protoni contro un bersaglio, essi vengono rallentati e non penetrano più di tanto. Al contrario i neutroni è possibile che passino indisturbati (sono molto penetranti). Se i neutroni collidono con un nucleo, normalmente avviene una reazione di knock-out, ossia esce una particella, tipicamente il protone.
    \item Quindi serve il vuoto nella camera dei neutroni, si mette polvere di polonio per generare particelle $\alpha$ e si mette un bersaglio di berillio.  Il neutrone penetra il bersaglio di berillio e arriva fino alla camera di ionizzazione. Inoltre si mette della paraffina tra bersaglio e camera a ionizzazione. La paraffina è ricca di protoni, faccio questo per massimizzare la sezione d'urto protone-neutrone, che è elevata perché hanno masse simili. Chiaramente nella camera a ionizzazione rivelo protoni e non neutroni.
\end{itemize}
Vediamo perché quella reazione è sbagliata.
\begin{itemize}
    \item La reazione sbagliata che hanno considerato i Curie sarebbe stata $ \alpha + ^9_4Be \rightarrow ^{13}_6C + \gamma$, che possiamo esprimerla come $\gamma + p \rightarrow \gamma + p$, cioè effetto Compton, ma con il protone.
    \item Rivediamo velocemente l'effetto Compton.
    %\begin{figure}[H]
    %    \centering
    %    \includegraphics[scale=0.5]{Compton.png}
    %    \caption{Effetto Compton.}
    %    \label{fig:Compton}
    %\end{figure}[H]
    \item Si ha dalla conservazione dell'impulso lungo le due direzioni
\end{itemize}
\begin{flalign*}
    & \begin{aligned} & \begin{cases}
        p_0=p_1\cos\vartheta+p\cos\varphi\\
        0 = p_1\sin\vartheta-p\sin\varphi
    \end{cases}\\
    %\MoveEqLeft[-1]\text{}
    \end{aligned}
    & &
    \begin{aligned}
        & \begin{cases}
        p_0^2+p_1^2\cos^2\vartheta-2p_0p_1\cos\vartheta=p^2\cos^2\varphi\\
        p_1^2\sin^2\vartheta=p^2\sin^2\varphi
      \end{cases} \\
      %\MoveEqLeft[-1]\text{si pentru legea a II-a}
    \end{aligned}
    & &
    \begin{aligned}
        & \begin{cases}
        p^2=p_0^2+p_1^2-2p_0p_1\cos\vartheta\\
        p_1^2\sin^2\vartheta=p^2\sin^2\varphi
      \end{cases} \\
      %\MoveEqLeft[-1]\text{si pentru legea a II-a}
    \end{aligned}
  \end{flalign*}
  \begin{itemize}
  \item Invece dalla conservazione dell'energia
  \begin{equation*}
      E_0+m_ec^2=E_1+T+m_ec^2\Rightarrow E_0-E_1=T\Rightarrow c(p_0-p_1)=T
  \end{equation*}
  \item Vogliamo l'energia dell'elettrone: $E_{\text{TOT}}^{\text{elettrone}}=m_ec^2+T=(m_e^2+c^4+c^2p^2)^{\frac12}$ (usando la relazione di mass-shell) ed elevando al quadrato troviamo $\frac{T^2}{c^2}+2m_eT=p_0^2+p_1^2-2p_0p_1\cos\vartheta$ da cui usando la conservazione dell'energia troviamo
  \begin{equation*}
    \frac{p_0-p_1}{p_0p_1}=\frac1{m_ec}(1-\cos\vartheta)\Rightarrow \frac{E_0-E_1}{E_0E_1}=\frac1{m_ec^2}(1-\cos\vartheta)
  \end{equation*}
  dove $E_0$ è l'energia del fotone incidente e $E_1$ è l'energia del fotone diffuso. Possiamo ottenere 
  \begin{equation*}
    E_1 = \frac{E_0}{1+\frac{E_0}{m_ec^2}(1-\cos\vartheta)}
  \end{equation*}
Adesso possiamo trovare l'energia dell'elettrone da $E=E_0-E_1$. Otteniamo
\begin{equation*}
    E=\frac{\frac{E_0^2}{m_ec^2}(1-\cos\vartheta)}{1+\frac{E_0}{m_ec^2}(1-\cos\vartheta)}\Rightarrow E_{\text{MAX}}^{\text{elettrone}}=\frac{\frac{2E_0^2}{m_ec^2}}{1+2\frac{E_0}{m_ec^2}}=\frac{2E_0^2}{m_ec^2+2E_0}
\end{equation*}
    \item Tornando a Curie, al posto della massa dell'elettrone mettiamo la massa del protone. Dall'ultima relazione otteniamo l'equazione di secondo grado in $E_0$
    \begin{equation*}
        2E_0^2-2E_0E_p^{\text{MAX}}-m_pc^2E_p^{\text{MAX}}=0
    \end{equation*}
    Avendo misurato $E_p^{\text{MAX}}\approx 5.3$ MeV, si ottiene $E_0\approx 52.6$ MeV. Finora negli esperimenti erano abituati a qualche MeV di energia per i fotoni, quindi era una novità. Tuttavia, il problema non era questo. Con l'interpretazione di Curie avrebbero $E_\gamma=m(\alpha)+m\qty(^9_4Be)-m\qty(^{13}C)\approx 11$ MeV. Invece con Chadwick si ha $T_n=m(\alpha)+m\qty(^9_4Be)-m\qty(^{12}C)- m(n)=945.3 MeV - m(n)$.
\item Dunque con Curie\footnote{Majorana nel frattempo li perculava dicendo che non sapevano di aver già scoperto il neutrone.} servirebbe un fotone di energia di circa $50$ MeV che è incompatibile con l'energia a disposizione di soli $11$ MeV. Invece se ho il neutrone, la sua energia cinetica sarà pari a $T_n\approx 945.3 MeV - m(n)$.
\end{itemize}
Ricaviamo la relazione che lega dati sperimentali con la massa del neutrone.
\begin{itemize}
    \item Stiamo considerando un urto elastico neutrone-protone, quindi si conserveranno sia energia ed impulso.
    \begin{equation}
        \begin{cases}
          m_nv^{\text{MAX}}_n=m_pu_p-m_nv_n'\\
          \frac12 m_nv^{\text{MAX,}2}_n=\frac12 m_pu_p^2+\frac12 m_nv_n'^2
        \end{cases}\,
    \end{equation}
    Si ricava $v_n'$ dalla prima e lo si sostituisce nella seconda. Alla fine si ottiene 
    \begin{equation*}
        u_p=\frac{2m_n}{m_n+m_p}v_n^{\text{MAX}}
    \end{equation*}
    Se ripetessimo l'esperimento con scattering di azoto, si trova la stessa formula ma con azoto $u_N=\frac{2m_n}{m_n+m_N}v_n^{\text{MAX}}$. 
    \item Noi misuriamo $u_p$ e $u_N$ mentre $m_N$ e $m_p$ sono note, quindi facendo il rapporto si trova 
    \begin{equation*}
        \frac{u_p}{u_N}=\frac{m_N+m_n}{m_p+m_n}\Rightarrow m_n\approx m_p
    \end{equation*}
    Anche se in realtà la massa del neutrone è leggermente superiore di quella del protone. 
    \item Abbiamo detto che si può misurare la velocità dal range della particella carica che attraversa il materiale. Questo è possibile grazie alla formula di Bethe-Bloch. 
\end{itemize}
\subsection{Derivazione formula di Bethe-Bloch}\label{subsec:bethe-bloch}
L'interazione radiazione-materia è alla base della rivelazione di particelle. Consideriamo solo particelle cariche.
\begin{itemize}
    \item Interagiscono il campo della particella incidente con il campo del mezzo mediante ionizzazione ed eccitazione. Così costruisco diverse tipologie di rivelatori. 
    \item Sono rilevanti il tipo di materiale, il tipo e la velocità della particella incidente. Anche se nel nostro caso non proviene da un fascio da ma una reazione. 
    \item Se $b>R\_{atomico}$ eccito e/o ionizzo e vedo l'atomo come un blocco. Non c'è deflessione.
    \item Se $b\approx R\_a$ uguale a prima ma l'elettrone atomico è come se fosse libero.
    \item Se $R\_N<b<R\_a$ vedo il nucleo e c'è forte deflessione.
    \item Le cariche possono: ionizzare, emettere luce di scintillazione, avere effetto Cherenkov o emettere radiazione di transizione. I $\gamma$ possono interagire con effetto fotoelettrico, effetto Compton o produzione di coppie. I neutroni urtano un nucleo che rincula. Infine i neutrini possono diffondersi solo per interazione debole, con $\sigma\approx10^{-41}$ cm$^2$ e quindi servono tonnellate di materiale per rivelarle. 
\end{itemize}
Sappiamo che la perdita di energia $\dv{E}{x}\propto z^2\cdot \frac1{\beta^2}\cdot \frac Z A$. 
\begin{itemize}
    \item Si usa di solito $\frac 1 \rho \dv{E}{x}$ per cui il \textit{MIP} (Minimum Ionizing Particle) è simile per tutti i materiali ed è circa $1-2\, \frac{\text{MeV g}}{\text{cm}^2}$.    
    \item Da $\dv{E}{x}$ si può ricavare il range residuo della particella carica. Supponendo che  $E_0$ perda energia solo da ionizzazione/eccitazione, possiamo esprimere il range come
    \begin{equation*}
        R=\int_0^R\dd{x}=\int_{E_0}^{mc^2} \qty(\dv{E}{x})^{-1}\dd{E}= m\cdot F(v)
    \end{equation*}
    quindi il range è funzione della velocità iniziale della particella. A quei tempi il range si misurava dalle camere a nebbia manualmente con un righello! E così si ottenevano informazioni sulla velocità della particella.
    \item Per le particelle neutre non abbiamo tracce, però possiamo guardare le cariche prodotte dalla collisione. Ad esempio se i prodotti sono $^{12}C+n$, mettiamo uno strato di paraffina così che dallo scattering $n-p$ riveliamo il protone 
\end{itemize}
