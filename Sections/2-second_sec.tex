Riprenderei appunti di Russo e basta. Riportiamo cmq quello che dice lei.
Ci sono quattro principi base nella relatività.
\begin{itemize}
    \item Ogni legge fisica è invariante in ogni sistema di riferimento inerziale.
    \item Energia, impulso, momento angolare in ogni sistema fisica isolato si conservano.
    \item La velocità della luce è la stessa in ogni sistema di riferimento.
    \item Il tempo non è invariante (assoluto).
    \item I primi due sono legati alla meccanica classica, gli ultimi due alla relatività. Da ciò ne segue che le trasformazioni galileiane non valgono più, e al loro posto ci sono le trasformazioni di Lorentz. Nel limite $\beta\ll1$ le trasformazioni di Lorentz diventano quelle di Galileo.
\end{itemize}
\subsection{Trasformazioni di Lorentz}
\begin{itemize}
    \item Definiamo i quadrivettori come $A=(a_0,a_i)=(a_0,\vec a)$, con $a_0$ componente temporale e $a_i$ componente spaziale.
    \item Definiamo il prodotto scalare tra quadrivettori come $\tilde A\tilde B =a_0b_0-a_ib_i$. Questo prodotto è invariante sotto trasformazioni di Lorentz.
    \item Se consideriamo un sistema di riferimento $S$ e un altro $S'$, con $S'$ che si muove rispetto a $S$ con velocità $v$ lungo l'asse $x$, allora le trasformazioni di Lorentz sono date da:
    \begin{equation*}
        \mqty(a'_0 \\\xmat*{a'}{3}{1})= 
        \underbrace{\begin{pmatrix}
            \mqty{\gamma & -\beta\gamma & 0 & 0 \\ -\beta\gamma & \gamma &0&0 \\ 
            0 & 0 & 1 & 0 \\ 
            0 & 0 & 0 & 1 }
        \end{pmatrix}}_{=L(\beta)}
        \mqty(a_0 \\\xmat*{a}{3}{1})= 
        \mqty(\gamma a_0-\beta\gamma a_1\\-\beta\gamma a_0+\gamma a_1\\a_2\\a_3)\implies 
        \begin{cases}
            ct'=\gamma(ct-\beta x) \\
            x'=\gamma(x-\beta ct)
        \end{cases}
    \end{equation*}
    Una proprietà importante è $L(\beta)^{-1}=L(-\beta)$, da cui ne segue che per invertire le trasformazioni di Lorentz basta scambiare variabile con indice con quelle senza e $\beta\to-\beta$.
    \item Che si ha al limite non relativistico? Supponiamo $\beta\ll1\implies\gamma\approx1+\frac{\beta^2}{2}\approx1$. Ne segue per il tempo che
    \begin{equation*}
        ct'=\qty(1+\frac{\beta^2}2)(ct-\beta x) \approx ct-\beta x + \frac{\beta^2}2ct\approx ct\implies t'=t 
    \end{equation*}
    e per lo spazio che 
    \begin{equation*}
        x'=\qty(1+\frac{\beta^2}2)(x-\beta ct)\approx x-\beta ct\implies x'= x - vt
    \end{equation*}
    \item Se il moto non è solo lungo $x$, allora dobbiamo considerare $\vec\beta=\frac{\vec v} c$ e 
    \begin{equation*}
        \begin{cases}
            ct'=\gamma(ct-\beta x_\parallel) \\
            x'_\parallel=\gamma(x_\parallel-\beta ct)\\
            \vec x_\perp'=\vec x_\perp
        \end{cases}
    \end{equation*}
    \item Dimostriamo che il prodotto scalare è invariante. Sia $A=(a_0,\vec a)$ e $B=(b_0,\vec b)$. Calcoliamo $A'\cdot B'$.
    \begin{equation*}
        A'\cdot B'=a_0'b_0'-\vec a'\cdot\vec b'=\gamma^2(a_0-\beta a_1)(b_0-\beta b_1)-\gamma^2(a_1-\beta a_0)(b_1-\beta b_0)-a_2b_2-a_3b_3=\dots=A\cdot B
    \end{equation*}
\end{itemize}
Vediamo alcune conseguenze in high energy physics.
\begin{itemize}
    \item La contrazione delle lunghezze. Consideriamo un oggetto di lunghezza $L$ che si muove con velocità $v$. Supponiamo che il sistema solidale ad esso sia $S'$ e la lunghezza misurata sia $d'=x_2'-x_1'$ che avviene simultaneamente quindi $t_2=t_1$. Se trasformiamo otteniamo $d'=x_2'-x_1'=\gamma(x_2-\beta ct_2)-\gamma(x_1-bct_1)=\gamma(x_2-x_1)-\gamma\beta c(t_2-t_1)=\gamma(x_2-x_1)=d\implies d'=\gamma d$. Ne segue che la lunghezza misurata da un osservatore in moto è contratta di un fattore $\gamma$, e la lunghezza propria, misurata nel sistema solidale all'oggetto è la massima possibile.
    \item La dilatazione temporale. Consideriamo due eventi che avvengono nello stesso punto nello spazio, ma in tempi diversi. Se trasformiamo otteniamo $c\Delta t=c(t_2-t_1)=\gamma c(t_2'-t_1')+\beta \gamma (x_2'-x_1')=\gamma c\Delta t'$. Ne segue che il tempo misurato da un osservatore in moto è dilatato di un fattore $\gamma$, e il tempo proprio, misurato nel sistema solidale all'oggetto è il minimo possibile. Da ciò si hanno varie conseguenze.
\end{itemize}
Slide su esperimento di Conversi, Pancini, Piccioni (CPP) sui pioni e muoni.
\begin{itemize}
    \item Nel 1912 Hess scoprì i raggi cosmici. Nel 1932 Anderson scoprì i positroni, predetti da Dirac nel 1928 (già discussa).
    \item Nel 1935 Yukawa introdusse la teoria delle interazioni forti, predicendo una massa mediatrice di $\sim 100$ MeV. Il mesone di Yukawa doveva decadere in elettrone e netruino con tempo di decadimento di $\sim1\mu$s. Nel 1937 si scoprì il mesotrone (Anderson e Neddermeyer), con una massa di $110$ MeV, associata alla particella di Yukawa. Nel 1940 si studiò assorbimento e decadimento delle proprietà di assorbimento del mesone di Yukawa.
    \item Il decadimento del mestrone (che in realtà è un $\mu$) fu studiato diverse volte. Nel 1940 si osservò il suo decadimento in positroni; nel 1941 ci fu una misura da Rasetti che ottenne $\tau=(1.5\pm0.3)\mu$s. Nel 1941 Piccioni e Conversi decisero di lavorare assieme e migliorare la precisione nella misura del tempo di decadimento (del mesone di Yukawa).
    \item Nel 1939 Montgomery fece un esperimento (\autoref{fig:montgomery}) per misurare il decadimento del $\mu$
    \begin{figure}[h]
        \centering
        \includegraphics[width=0.4\textwidth]{immagini/fig_montgomery.png}
        \caption{Esperimento di Montgomery. Volevano estrarre il tempo di decadimento dalle intensità delle coincidenze ritardate con e senza stopper. Purtroppo non riuscirono a misurare il tempo di decadimento del $\mu$ a causa del troppo rumore in B.}
        \label{fig:montgomery}
    \end{figure}
    \item Un altro tentativo fu fatto da Rasetti nel 1940, con un apparato più complicato (\autoref{fig:rasetti}).
    \begin{figure}[h]
        \centering
        \includegraphics[width=0.6\textwidth]{immagini/fig_rasetti.png}
        \caption{Disposizione dei contatori, illustrando le connessioni con gli amplificatori.}
        \label{fig:rasetti}
      \end{figure}
    Nella procedura sperimentale si definisce un fascio di mestroni con la coincidenza ABCD. L'anticontatore G discrimina dagli sciami elettromagnetici. L'anticontatore F seleziona i mestroni che si sono fermati nell'assorbitore. Il contatore E rivela particelle emesse nell'assorbitore. Non si usarono coincidenze ritardate ma "immediate" con tempi di risoluzione diversi. Guardando le combinazioni dei tempi con cui il segnale arriva, hanno fatto un fit particolare ed estratto il tempo di decadimento. Ottennero $\tau=(1.5\pm0.3)\mu$s.
    \item Sucessivamente l'esperimento fu riproposto da Conversi, Pancini e Piccioni con l'idea di effettuare una misura migliore (vedi \autoref{fig:cpp})
    \begin{figure}[h]
        \centering
        \includegraphics[width=0.4\textwidth]{immagini/fig_cpp.png}
        \caption{}
        \label{fig:cpp}
      \end{figure}
    Un mesotrone si ferma nell'assorbitore (Fe) e poi decade. Si usano coincidenze ritardate tra i contatori sopra e sotto e le anticoincidenze (A). Gli elettroni si fermano nell'assorbitore di Pb. Le anticoincidenze servono a scartare l'evento. Misurarono $\tau=(2.33\pm0.15)\mu$s, meglio di Rasetti. 
\end{itemize}
Esercizi di cinematica relativistica.
\begin{itemize}
    \item $m_\mu=106$ MeV, $\tau_\mu=2.2\cdot10^{-6}$s, $p_\mu=10$ GeV, $l=10km$. Calcolare la probabilità di avere il muone sulla superficie nel riferimento del laboratorio e in quello solidale al muone. Si parte da $\dv{\probP}{t}=\frac1\tau e^{-t/\tau}\implies \frac1{\tau\beta c}e^{-x/\beta c\tau}$, con $t=\frac x{\beta c}$. Allora si ottiene  (ricordiamo $\beta\gamma c=\frac p m$)
    \begin{equation*}
    \probP=\int_0^{d/\gamma}\frac1{\tau\beta c}e^{-x/\beta c \tau}\dd{x}=\dots =0.85
    \end{equation*}
    85\% è la probabilità che il muone arrivi sulla superficie prima di decadere nel riferimento solidale. Classicamente non è possibile. Invece nel riferimento di laboratorio si fa lo stesso calcolo ma con tempo dilatato e lunghezza non contratta. Si ottiene esattamente lo stesso valore, questo è dovuto al fatto che la probabilità è indipendente dal sistema di riferimento.
    \item Stesso principio dello scorso esercizio, si usa in fisica degli acceleratori quando serve trasportare per lunghe distanze i fasci prima di collidere contro il bersaglio. $m_\pi=140$ MeV, $\tau_\pi=2.56\cdot10^{-8}$s, $p_\pi=200$ GeV. Il limite galileiano $c\tau\approx 8$m, poco rispetto ai 300 m del nostro problema. La $\probP\_{classica}=e^{-\frac{ct}{c\tau}}\sim 10^{-17}$ bassissima. Invece per quella relativistica ricaviamo $\gamma=\frac E {m}=1429$. Ricaviamo il $\tau\_{lab}=\gamma \tau =3.71\cdot 10^{-5}$s e il tempo da attraversare $t=l=10^{-6}$s. Allora la probabilità vale $\probP\_{rel}=e^{-t/\tau\_{lab}}=0.97$. Se volessi la distanza $d=\beta \gamma \tau=11$km.
    \item Torniamo ai quadrivettori. La matrice della metrica la conosciamo è $g^{\mu\nu}=\text{diag}(1,-1,-1,-1)$. Un quadrivettore importante è quello di impulso $P^\mu=(E,\vec p)$. Questo quadrivettore non è invariante ma il suo quadrato sì. Facendolo, scopriamo che vale $P^\mu P_\mu=m^2$ usando la relazione di mass-shell.
    \item Alcune relazioni utili sono: $p=\gamma mv$, $E=\gamma m$, $T=mc^2(\gamma-1)$, $\beta=\frac p E$.
    \item Quando facciamo esperimento di fisica di particelle, ci mettiamo nel riferimento di laboratorio solidale all'osservatore e ai rivelatori. In questo riferimento tipicamente $v\_T=0$. Possiamo scrivere allora nel riferimento del laboratorio $P_1=(E_1,\vec {p}_1)$, $P_2 = (m\_T,0)$ (1 = proiettile, 2 = target). Allora $P\_{tot}=sP_1+P_2=(E_1+m\_T,\vec p_1)$. 
    \item Un altro sistema di riferimento utile è quello del centro di massa, in cui $\vec {p}\_{tot}=0$. Se consideriamo una particella incidente in un target, avremo $P_1^*=(E_1^*,\vec p)$ e $P_2^*=(E_2^*,-\vec p)$. Allora $P\_{tot}^*=(E_1^*+E_2^*,0)$. Poichè è un invariante, $P\_{tot}^2$ sarà uguale nel riferimento del centro di massa e in quell del laboratorio. Ne segue che $P\_{tot}^{*,2}=E^2\_{CM}=P\_{tot}^{\text{lab},2}=(E_1+m\_T)^2-p_1^2=E_1^2+m\_T^2+2E_1m\_T-p_1^2=m_1^2+2E_1m\_T+m\_T^2$. ('sta roba è inutile?)
    \item Se consideriamo un fascio di $n$ particelle allora dovremo considerare la sommatoria. $P_\mu P^\mu= (E_1+E_2+\dots+E_n)^2-(\vec p_1+\vec p_2+\dots+\vec p_n)^2$ in generale. Se consideriamo il sistema del centro di massa avremo $P^{\mu,*}=(E\_{CM},0)$. Quindi nel riferimento del centro di massa questo scalare è solo il quadrato dell'energia nel sistema del centro di massa.
    \item Riprendiamo la storia. Nel 1947 ci fu la scoperta del pione, usando le emulsioni nucleari. Decade in sequenza il pione in muone ed infine in elettrone. In tutti gli eventi il muone aveva energia di 4.1 MeV a cui corrisponde il range misurato di 600 $\mu$m, le strisce sono tutte di uguale lunghezza (sappiamo da Bethe-Bloch che il range e l'impulso sono legati). In base a densità di ionizzazione ($\dd{E}/\dd{x}$) distinguiamo il tipo di particella.
    \item Visto che il range è sempre lo stesso, l'impulso del muone sarà sempre lo stesso 29 MeV. Visto che l'impulso è sempre lo stesso, il decadimento deve essere a due corpi (e non tre altrimenti si avrebbe spettro continuo). 
    \item Supponendo che si abbia un decadimento a due corpi con la particella iniziale che si ferma, abbiamo $X\to\mu+\nu_\mu$ e il sistema del centro di massa coincide con il sistema solidale a $X$ (il $\pi$). Da ciò ne segue che impulso di neutrino e muone sono uguali ed opposti, quindi  
    \begin{equation*}
    m\_X^2=E\_{cm}^2=(E_\mu+E_\nu)^2\implies m\_X=\sqrt{m_\mu^2+p_\mu^2}+p_\nu=\dots=138.9 \text{ MeV}
    \end{equation*}
    
\end{itemize}